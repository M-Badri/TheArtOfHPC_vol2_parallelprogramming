% -*- latex -*-
%%%%%%%%%%%%%%%%%%%%%%%%%%%%%%%%
%%%%%%%%%%%%%%%%%%%%%%%%%%%%%%%%
%%
%% This text file is part of the source of 
%% `Parallel Computing'
%% by Victor Eijkhout, copyright 2012-2022
%%
%% MPI API file for MPI_Type_get_contents
%%
%% THIS FILE IS AUTO-GENERATED
%%
%%%%%%%%%%%%%%%%%%%%%%%%%%%%%%%%
%%%%%%%%%%%%%%%%%%%%%%%%%%%%%%%%

\begingroup
\ttfamily\bfseries
\catcode`\_=12
\begin{tabular}{
l       % name
l       % param name
p{\mpiparmtextsize}  % explanation
p{.85in} % ctype
p{.9in} % ftype
l       % inout
}
\toprule
\mdseries\textrm{Name}&
\mdseries\textrm{Param name}&
\mdseries\textrm{Explanation}&
\mdseries\textrm{C type}&
\mdseries\textrm{F type}&
\mdseries\textrm{inout}\\
\midrule
\hbox to 18pt{MPI_Type_get_contents (\hss} \\
\hbox to 18pt{MPI_Type_get_contents_c (\hss} \\
 & datatype & datatype to decode & MPI_Datatype & TYPE\discretionary{}{\kern10pt}{}(MPI_Datatype)  & IN \\
 & max_integers & number of elements in \mpiarg{array_of_integers} & $\left[ \begin{array}{ll}\mathtt{ int }  \\ \mathtt{ MPI_Count} \end{array} \right.$  & INTEGER  & IN \\ [+3pt]
 & max_addresses & number of elements in \mpiarg{array_of_addresses} & $\left[ \begin{array}{ll}\mathtt{ int }  \\ \mathtt{ MPI_Count} \end{array} \right.$  & INTEGER  & IN \\ [+3pt]
 & max_datatypes & number of elements in \mpiarg{array_of_large_counts} & $\left[ \begin{array}{ll}\mathtt{ int }  \\ \mathtt{ MPI_Count} \end{array} \right.$  & INTEGER  & IN \\ [+3pt]
 & array_of_integers & number of elements in \mpiarg{array_of_datatypes} & $\left[ \begin{array}{ll}\mathtt{ int[] }  \\ \mathtt{ MPI_Count} \end{array} \right.$  & INTEGER\discretionary{}{\kern10pt}{}(max_integers)  & IN \\ [+3pt]
 & array_of_addresses & contains integer arguments used in constructing \mpiarg{datatype} & $\left[ \begin{array}{ll}\mathtt{ MPI_Aint[] }  \\ \mathtt{ int[]} \end{array} \right.$  & INTEGER\discretionary{}{\kern10pt}{}(KIND=MPI_ADDRESS_KIND)\discretionary{}{\kern10pt}{}(max_addresses)  & OUT \\ [+3pt]
 & array_of_datatypes & contains address arguments used in constructing \mpiarg{datatype} & $\left[ \begin{array}{ll}\mathtt{ MPI_Datatype[] }  \\ \mathtt{ MPI_Aint[]} \end{array} \right.$  & TYPE\discretionary{}{\kern10pt}{}(MPI_Datatype)\discretionary{}{\kern10pt}{}(max_datatypes)  & OUT \\ [+3pt]

&)\\

\bottomrule
\end{tabular}
\endgroup

