% -*- latex -*-
%%%%%%%%%%%%%%%%%%%%%%%%%%%%%%%%%%%%%%%%%%%%%%%%%%%%%%%%%%%%%%%%
%%%%%%%%%%%%%%%%%%%%%%%%%%%%%%%%%%%%%%%%%%%%%%%%%%%%%%%%%%%%%%%%
%%%%
%%%% This text file is part of the source of
%%%% `Parallel Programming in MPI and OpenMP'
%%%% by Victor Eijkhout, copyright 2012-9
%%%%
%%%% architecture.tex : about computer architecture
%%%%
%%%%%%%%%%%%%%%%%%%%%%%%%%%%%%%%%%%%%%%%%%%%%%%%%%%%%%%%%%%%%%%%
%%%%%%%%%%%%%%%%%%%%%%%%%%%%%%%%%%%%%%%%%%%%%%%%%%%%%%%%%%%%%%%%

There is much that can be said about computer architecture. However,
in the context of parallel programming we are mostly concerned with
the following:
\begin{itemize}
\item How many networked nodes are there, and does the network have a
  structure that we need to pay attention to?
\item On a compute node, how many sockets (or other \ac{NUMA} domains)
  are there?
\item For each socket, how many cores and hyperthreads are there? Are
  caches shared?
\end{itemize}

\Level 0 {Tools for discovery}

An easy way for discovering the structure of your parallel machine is
to use tools that are written especially for this purpose.

\Level 1 {Intel cpuinfo}

The \indextermbus{Intel}{compiler suite} comes with a tool
\indextermdef{cpuinfo} that reports on the structure of the node
you are running on. It reports on the number of \indexterm{package}s,
that is: sockets, cores, and threads.

\Level 1 {hwloc}

The open source package \indextermtt{hwloc} does similar reporting to
cpuinfo, but it has been ported to many platforms. Additionally, it
can generate ascii and pdf graphic renderings of the architecture.
