% -*- latex -*-
%%%%%%%%%%%%%%%%%%%%%%%%%%%%%%%%%%%%%%%%%%%%%%%%%%%%%%%%%%%%%%%%
%%%%%%%%%%%%%%%%%%%%%%%%%%%%%%%%%%%%%%%%%%%%%%%%%%%%%%%%%%%%%%%%
%%%%
%%%% This text file is part of the source of 
%%%% `Parallel Programming in MPI and OpenMP'
%%%% by Victor Eijkhout, copyright 2012-2022
%%%%
%%%% mpi-rank.tex : comm size and rank and such
%%%%
%%%%%%%%%%%%%%%%%%%%%%%%%%%%%%%%%%%%%%%%%%%%%%%%%%%%%%%%%%%%%%%%
%%%%%%%%%%%%%%%%%%%%%%%%%%%%%%%%%%%%%%%%%%%%%%%%%%%%%%%%%%%%%%%%

\Level 0 {Processor identification}
\label{sec:rank-size}

Since all processes in an MPI job are instantiations of the same executable,
you'd think that they all execute the exact same instructions,
which would not be terribly useful.
You will now learn how to distinguish
processes from each other, so that together they can start doing
useful work.

\Level 1 {Processor name}

In the following exercise you will print out the hostname 
of each MPI process
%
with
\indexmpiref{MPI_Get_processor_name}
%
as a first way of distinguishing between processes.
This routine has a character buffer argument, which
needs to be allocated by you.
The length of the buffer is also passed,
and on return that parameter has the actually used length.
The maximum needed length is \indexmpishow{MPI_MAX_PROCESSOR_NAME}.
%
\csnippetwithoutput{procname}{examples/mpi/c}{procname}
%
(Underallocating the buffer will not lead to a runtime error.)

\begin{fortrannote}{Processor name}
  Allocate a \lstinline{Character} variable with
  the appropriate length.
  The returned value of the length parameter can assist 
  in printing the result:

  \fsnippetwithoutput{procnamef}{examples/mpi/f08}{procname}
\end{fortrannote}

\begin{exercise}
  \label{ex:procname}
  Use the command \indexmpishow{MPI_Get_processor_name}.
  Confirm that you are able to run a program that uses two different nodes.

\begin{tacc}
    TACC nodes have a hostname \n{cRRR-CNN}, where RRR is the rack number, C is the chassis
    number in the rack, and NN is the node number within the chassis. Communication
    is faster inside a rack than between racks!
\end{tacc}
\end{exercise}

\begin{mplnote}{Processor name}
  The \indexmplshow{processor_name} call is an environment method
  returning a \lstinline{std::string}:
\begin{lstlisting}
std::string mpl::environment::processor_name ();
\end{lstlisting}
\end{mplnote}

\Level 1 {Communicators}

First we need to introduce the concept of
\indextermdef{communicator}, which is an abstract description of a
group of processes. For now you only need to know about the existence
of the \indexmpishow{MPI_Comm} data type, and that there is a
pre-defined communicator \indexmpishow{MPI_COMM_WORLD} which
describes all the processes involved in your parallel run.


In the procedural languages~C,
a \emph{communicator}\index{communicator!variable} is a \emph{variable}
that is passed to most routines:
\begin{lstlisting}
#include <mpi.h>
MPI_Comm comm = MPI_COMM_WORLD;
MPI_Send( /* stuff */ comm );
\end{lstlisting}

\begin{fortrannote}{Communicator type}
  In Fortran, pre-2008 a communicator was an \indexterm{opaque handle},
  stored in an \lstinline{Integer}. With \indextermbus{Fortran}{2008},
  communicators are derived types:
\begin{lstlisting}
use mpi_f098
Type(MPI_Comm} :: comm = MPI_COMM_WORLD
call MPI_Send( ... comm )
\end{lstlisting}
\end{fortrannote}

\begin{pythonnote}{Communicator objects}
In object-oriented languages, a communicator is an
\emph{object}\index{communicator!object},
and rather than passing it to routines,
the routines are often methods of the communicator object:
\begin{lstlisting}
from mpi4py import MPI
comm = MPI.COMM_WORLD
comm.Send( buffer, target )
\end{lstlisting}
\end{pythonnote}

\begin{mplnote}{World communicator}
  The naive way of declaring a communicator would be:
  %
  \cxxverbatimsnippet[examples/mpi/mpl/commrank.cxx]{mplcommcopy}
  %
  calling the predefined environment method \indexmplshow{comm_world}.

  However, if the variable will always correspond to the world communicator,
  it is better to make it \lstinline{const} and declare it to be a reference:
  %
  \cxxverbatimsnippet[examples/mpi/mpl/commrank.cxx]{mplcommref}
\end{mplnote}

\begin{mplnote}{Communicator copying}
  The communicator class has its copy operator deleted;
  however, copy initialization exists:
  %
  \cxxverbatimsnippet[examples/mpi/mpl/commcompare.cxx]{mplcompare}
  %
  (This outputs true/false/false respectively.)

  \begin{mplimpl}
    The copy initializer performs an \indexmpishow{MPI_Comm_dup}.
  \end{mplimpl}
\end{mplnote}

\begin{mplnote}{Communicator passing}
  Pass communicators by reference to avoid communicator duplication:
  %
  \cxxverbatimsnippet{mplcommpass}
\end{mplnote}

You will
learn much more about communicators in chapter~\ref{ch:mpi-comm}.

\Level 1 {Process and communicator properties: rank and size}

To distinguish between processes in a communicator, MPI provides two calls
\begin{enumerate}
\item \indexmpiref{MPI_Comm_size} reports how many processes there are in all; and
\item \indexmpiref{MPI_Comm_rank} states what the number of the
  process is that calls this routine.
\end{enumerate}

If every process executes the \indexmpishow{MPI_Comm_size} call, they all get the
same result, namely the total number of processes in your run.
%
On the
other hand, if every process executes \indexmpishow{MPI_Comm_rank}, they all get
a different result, namely their own unique number, an integer in the
range from zero to the number of processes minus~1.
See figure~\ref{fig:rank-parallel}.
%
\begin{figure}[ht]
  \includegraphics[scale=.5]{rank-parallel}
  \caption{Parallel program that prints process rank}
  \label{fig:rank-parallel}
\end{figure}
%
In other words, each process can find out `I~am process~5
out of a total of~20'.

%pyinput ex-rank.txt

\begin{exercise}
  \label{ex:hello3}
  Write a program where each process prints out a message
  reporting its number, and how many processes there are:
\begin{verbatim}
Hello from process 2 out of 5!
\end{verbatim}

  Write a second version of this program, where each process opens a
  unique file and writes to it. \emph{On some clusters this may not be
    advisable if you have large numbers of processors, since it can
    overload the file system.}
  \skeleton{commrank}
\end{exercise}
\begin{exercise}
  \label{ex:hello4}
  Write a program where only the process with number zero
  reports on how many processes there are in total.
\end{exercise}

In object-oriented approaches to MPI,
that is, \indextermtt{mpi4py} and \indexac{MPL},
the \indexmpishow{MPI_Comm_rank} and \indexmpishow{MPI_Comm_size}
routines are methods of the communicator class:

\begin{pythonnote}{Communicator rank and size}
  Rank and size are methods of the communicator object.
  Note that their names are slightly different from the MPI standard names.
\begin{lstlisting}
comm = MPI.COMM_WORLD
procid = comm.Get_rank()
nprocs = comm.Get_size()
\end{lstlisting}
\end{pythonnote}

\begin{mplnote}{Rank and size}
  The rank of a process (by \lstinline+mpl::communicator::+\indexmpldef{rank}) and
  the size of a communicator (by \lstinline+mpl::communicator::+\indexmpldef{size})
  are both methods of the \indexmplshow{communicator} class:
\begin{lstlisting}
const mpl::communicator &comm_world =
    mpl::environment::comm_world();
int procid = comm_world.rank();
int nprocs = comm_world.size();
\end{lstlisting}
\end{mplnote}
