% -*- latex -*-
%%%%%%%%%%%%%%%%%%%%%%%%%%%%%%%%%%%%%%%%%%%%%%%%%%%%%%%%%%%%%%%%
%%%%%%%%%%%%%%%%%%%%%%%%%%%%%%%%%%%%%%%%%%%%%%%%%%%%%%%%%%%%%%%%
%%%%
%%%% This text file is part of the source of
%%%% `Parallel Programming in MPI and OpenMP'
%%%% by Victor Eijkhout, copyright 2012-2023
%%%%
%%%% mpi-commsplit.tex : about splitting communicators
%%%%
%%%%%%%%%%%%%%%%%%%%%%%%%%%%%%%%%%%%%%%%%%%%%%%%%%%%%%%%%%%%%%%%
%%%%%%%%%%%%%%%%%%%%%%%%%%%%%%%%%%%%%%%%%%%%%%%%%%%%%%%%%%%%%%%%

\Level 0 {Communicators and groups}
\label{sec:mpi-comm-group}

You saw in section~\ref{sec:comm-split} that it is possible derive
communicators that have a subset of the processes of another communicator.
There is a more general mechanism, using \indexmpidef{MPI_Group}
objects.

Using groups, it takes the following steps to create a new communicator:
\begin{enumerate}
\item Access the \indexmpishow{MPI_Group} of a communicator
  object using \indexmpiref{MPI_Comm_group}.
\item Use various routines, discussed next, to form a new group.

  Note: you would form that group even on the processes that will not
  be come part of the new communicator.
\item Make a new communicator object from the group with using
  \indexmpiref{MPI_Comm_create}, collective on the old communicator.
\item On the ranks that were not in the subgroup, the resulting
  communicator value will be \indexmpishow{MPI_COMM_NULL}.
\end{enumerate}

There is also a routine \indexmpidef{MPI_Comm_create_group} that only
needs to be called on the group that constitutes the new communicator.

\Level 1 {Process groups}
\label{sec:comm-group}

Groups are manipulated with
\indexmpiref{MPI_Group_incl},
\indexmpiref{MPI_Group_excl},
\indexmpishow{MPI_Group_difference} and a few more.

\begin{lstlisting}
MPI_Comm_group (comm, group)
MPI_Comm_create (MPI_Comm comm,MPI_Group group, MPI_Comm newcomm)
\end{lstlisting}

\begin{lstlisting}
MPI_Group_union(group1, group2, newgroup)
MPI_Group_intersection(group1, group2, newgroup)
MPI_Group_difference(group1, group2, newgroup)
\end{lstlisting}

\begin{lstlisting}
MPI_Group_size(group, size)
MPI_Group_rank(group, rank)
\end{lstlisting}

Certain MPI types, \indexmpishow{MPI_Win} and \indexmpishow{MPI_File},
are created on a communicator.
While you can not directly extract that communicator from the object,
you can get the group with
\indexmpishow{MPI_Win_get_group} and \indexmpishow{MPI_File_get_group}.

There is a pre-defined empty group \indexmpidef{MPI_GROUP_EMPTY},
which can be used as an input to group construction routines,
or appear as the result of such operations as a zero intersection.
This not the same as \indexmpidef{MPI_GROUP_NULL},
which is the output of invalid operations on groups,
or the result of \indexmpishow{MPI_Group_free}.

\begin{mplnote}{Raw group handles}
  Should you need the \indexmpishow{MPI_Datatype} object
  contained in an MPL group,
  there is an access function \indexmplshow{native_handle}.
\end{mplnote}

\Level 1 {Examples}

Suppose you want to split the world communicator into
one manager process, with the remaining processes workers.
%
\cverbatimsnippet{groupworker}

\begin{exercise}[subgroup]
  \label{ex:comm-saling}
  Write a code that does a scaling study: your code needs to
  contain a loop over increasingly sized subsets of \indexmpishow{MPI_COMM_WORLD}.
  \begin{lstlisting}
    for (int subsize=1; subsize<=worldsize; subsize++) {
      MPI_Comm subcomm;
      // form `subcomm' to be of size `subsize'
      MPI_Allreduce( /* stuff */ subcomm );
      }
  \end{lstlisting}
  Carefully address which process do the various communicator and group calls;
  in particular do \indexmpishow{MPI_Comm_free} and \indexmpishow{MPI_Group_free}
  on the right processes.
\end{exercise}
