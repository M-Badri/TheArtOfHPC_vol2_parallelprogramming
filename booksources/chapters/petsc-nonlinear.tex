% -*- latex -*-
%%%%%%%%%%%%%%%%%%%%%%%%%%%%%%%%%%%%%%%%%%%%%%%%%%%%%%%%%%%%%%%%
%%%%%%%%%%%%%%%%%%%%%%%%%%%%%%%%%%%%%%%%%%%%%%%%%%%%%%%%%%%%%%%%
%%%%
%%%% This text file is part of the source of
%%%% `Parallel Programming in MPI and OpenMP'
%%%% by Victor Eijkhout, copyright 2012-2021
%%%%
%%%% petsc-nonlinear.tex : petsc nonlinear and time-dependent solvers
%%%%
%%%%%%%%%%%%%%%%%%%%%%%%%%%%%%%%%%%%%%%%%%%%%%%%%%%%%%%%%%%%%%%%
%%%%%%%%%%%%%%%%%%%%%%%%%%%%%%%%%%%%%%%%%%%%%%%%%%%%%%%%%%%%%%%%

\Level 0 {Nonlinear systems}

Nonlinear system solving means finding the zero of a general nonlinear function,
that is:
\[ \mathop{?}_x\colon f(x)=0 \]
with $f\colon \bbR^n-\bbR^n$.
In the special case of a linear function,
\[ f(x) = Ax-b, \]
we solve this by any of the methods in chapter~\ref{ch:petsc-solver}.

The general case can be solved by a number of methods,
foremost \indextermdef{Newton's method},
which iterates
\[ x_{n+1} = x_n - F(x_n)\inv f(x_n) \]
where $F$ is the \indexterm{Hessian} $F_{ij}=\partial f_i/\partial x_j$.

You see that you need to specify two functions
that are dependent on your specific problem:
the objective function itself, and its Hessian.

\Level 1 {Basic setup}

The PETSc nonlinear solver object is of type \indexpetscdef{SNES}:
`simple nonlinear equation solver'.
As with linear solvers, we create this solver on a communicator,
set its type, incorporate options,
and call the solution routine \indexpetscref{SNESSolve}:

\begin{lstlisting}
Vec value_vector,solution_vector;
/* vector creation code missing */
SNES solver;
SNESCreate( comm,&solver );
SNESSetFunction( solver,value_vector,formfunction, NULL );
SNESSetFromOptions( solver );
SNESSolve( solver,NULL,solution_vector );
\end{lstlisting}

The function has the type
\begin{lstlisting}
PetscErrorCode formfunction(SNES,Vec,Vec,void*)
\end{lstlisting}
where the parameters are:
\begin{itemize}
\item the solver object, so that you can access to its internal parameters
\item the $x$ value at which to evaluate the function
\item the result vector $f(x)$ for the given input
\item a context pointer for further application-specific information.
\end{itemize}

Example:
\begin{lstlisting}
PetscErrorCode evaluation_function( SNES solver,Vec x,Vec fx, void *ctx ) {
  const PetscReal *x_array;
  PetscReal *fx_array;
  VecGetArrayRead(fx,&fx_array);
  VecGetArray(x,&x_array);
  for (int i=0; i<localsize; i++)
    fx_array[i] = pointfunction( x_array[i] );
  VecRestoreArrayRead(fx,&fx_array);
  VecRestoreArray(x,&x_array);
};
\end{lstlisting}

Comparing the above to the introductory description you see that
the Hessian is not specified here.
An analytic Hessian can be dispensed with if you instruct
PETSc to approximate it by finite differences:
\[ H(x)y \approx \frac{f(x+hy)-f(x)}{h} \]
with $h$ some finite diference.
The commandline option \indexpetscoption{snes_fd} forces
the use of this finite difference approximation.
However, it may lead to a large number of function evaluations.
The option \indexpetscoption{snes_fd_color} applies a coloring
to the variables, leading to a drastic reduction in the number
of function evaluations.

If you can form the analytic Jacobian~/ Hessian, you can specify it with
\indexpetscref{SNESSetJacobian},
where the Jacobian is a function of type
\indexpetscref{SNESJacobianFunction}.

Specifying the Jacobian:
\begin{lstlisting}
Mat J;
ierr = MatCreate(comm,&J); CHKERRQ(ierr);
ierr = MatSetType(J,MATSEQDENSE); CHKERRQ(ierr);
ierr = MatSetSizes(J,n,n,N,N); CHKERRQ(ierr);
ierr = MatSetUp(J); CHKERRQ(ierr);
ierr = SNESSetJacobian(solver,J,J,&Jacobian,NULL); CHKERRQ(ierr);
\end{lstlisting}

\Level 0 {Time-stepping}

For cases
\[ u_t = G(t,u) \]
call \indexpetscdef{TSSetRHSFunction}.
\begin{lstlisting}
#include "petscts.h"  
PetscErrorCode TSSetRHSFunction
   (TS ts,Vec r,
    PetscErrorCode (*f)(TS,PetscReal,Vec,Vec,void*),
    void *ctx);
\end{lstlisting}

For implicit cases
\[ F( t,u,u_t ) = 0 \]
call \indexpetscdef{TSSetIFunction}
\begin{lstlisting}
#include "petscts.h"  
PetscErrorCode TSSetIFunction
   (TS ts,Vec r,TSIFunction f,void *ctx)
\end{lstlisting}
