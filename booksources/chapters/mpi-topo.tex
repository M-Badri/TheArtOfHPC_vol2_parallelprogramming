% -*- latex -*-
%%%%%%%%%%%%%%%%%%%%%%%%%%%%%%%%%%%%%%%%%%%%%%%%%%%%%%%%%%%%%%%%
%%%%%%%%%%%%%%%%%%%%%%%%%%%%%%%%%%%%%%%%%%%%%%%%%%%%%%%%%%%%%%%%
%%%%
%%%% This text file is part of the source of
%%%% `Parallel Programming in MPI and OpenMP'
%%%% by Victor Eijkhout, copyright 2012-2023
%%%%
%%%% mpi-topo.tex : about communicator topologies
%%%%
%%%%%%%%%%%%%%%%%%%%%%%%%%%%%%%%%%%%%%%%%%%%%%%%%%%%%%%%%%%%%%%%
%%%%%%%%%%%%%%%%%%%%%%%%%%%%%%%%%%%%%%%%%%%%%%%%%%%%%%%%%%%%%%%%

A communicator describes a group of processes, but the structure of
your computation may not be such that every process will communicate
with every other process. For instance, in a computation that is
mathematically defined on a
Cartesian 2D grid, the
processes themselves act as if they are two-dimensionally ordered and communicate
with N/S/E/W neighbors. If MPI had this knowledge about your
application, it could conceivably optimize for it, for instance by
renumbering the ranks so that communicating processes are closer
together physically in your cluster.

The mechanism to declare this structure of a computation to MPI
is known as a \indextermsub{virtual}{topology}. The following types of
topology are defined:
\begin{itemize}
\item \indexmpishow{MPI_UNDEFINED}: this values holds for communicators where no
  topology has explicitly been specified.
\item \indexmpidef{MPI_CART}: this value holds for Cartesian
  toppologies, where processes act as if they are ordered in a
  multi-dimensional `brick'; see
  section~\ref{sec:cartesian}.
\item \indexmpidef{MPI_GRAPH}: this value describes the graph
  topology that was defined in \mpistandard{1};
  section~\ref{sec:mpi-1-graph}. It is unnecessarily burdensome, since
  each process needs to know the total graph, and should therefore be
  considered obsolete; the type \indexmpishow{MPI_DIST_GRAPH} should
  be used instead.
\item \indexmpidef{MPI_DIST_GRAPH}: this value describes the distributed graph
  topology where each process only describes the edges in the process
  graph that touch itself; see section~\ref{sec:mpi-dist-graph}.
\end{itemize}
These values can be discovered with the routine
\indexmpishow{MPI_Topo_test}.

\Level 0 {Cartesian grid topology}
\label{sec:cartesian}

A \indextermsub{Cartesian}{grid} is a structure, typically in 2~or~3 dimensions,
of points that have two neighbors in each of the dimensions.
Thus, if a Cartesian grid has sizes $K\times M\times N$, its
points have coordinates $(k,m,n)$ with $0\leq k<K$ et cetera.
Most points have six neighbors $(k\pm1,m,n)$, $(k,m\pm1,n)$, $(k,m,n\pm1)$;
the exception are the edge points. A~grid where edge points
are connected through \indexterm{wraparound connections} is called
a \indextermsub{periodic}{grid}.

MPI has a `Cartesian communicator' construct
(that is, a communicator with type \indexmpishow{MPI_CART};
see above)
for processes to be organized not just linearly
through their ranks,
but also as if they are organized in a Cartesian grid.

\begin{mplnote}{Cartesian communicator}
  There is a separate class \indexmplshow{cartesian_communicator}.
  
  Additionaly, there is a class \indexmplshow{dimensions}
  that describes the shape of the Cartesian grid
  and its periodicity.
  The \lstinline{size(int)} method retrieves the size
  in the given dimension.
\end{mplnote}

The auxiliary routine \indexmpidef{MPI_Dims_create}
assists in finding a grid of a given dimension,
attempting to minimize the diameter.
%
%%\cverbatimsnippet{dimscreate}
{
  \def\snippetcodefraction{.45}
  \def\snippetlistfraction{.55}
  \csnippetwithoutput{dimscreate}{examples/mpi/c}{cartdims}
  }
%
If the dimensions array is nonzero in a component, that one is not touched. 
Of course, the product of the specified dimensions has to divide in
the input number of nodes.

\begin{mplnote}{Dims create}
  The \indexmplshow{dims_create} routine takes a \indexmplshow{dimensions}
  object with only the dimensionality specified, and creates one
  with the sizes filled in.
  \cxxverbatimsnippet{mpldimscreate}
  To have certain dimensions be periodic,
  the initial \indexmplshow{dimensions} object needs to be
  create with periodicity values \indexmplshow{periodic} or \indexmplshow{non_periodic}.
  \cxxverbatimsnippet{mpldimscreatep}
\end{mplnote}

\Level 1 {Cartesian topology communicator}
\label{sec:cart-comm-create}

The cartesian topology is specified by giving
\indexmpiref{MPI_Cart_create} the sizes of the processor grid along
each axis, and whether the grid is periodic along that axis.

\cverbatimsnippet{cartcreate}

(The Cartesian grid can have fewer processes than the input communicator:
any processes not included get \indexmpishow{MPI_COMM_NULL} as output.)

\begin{mplnote}{Cartesian communicator create}
  The actual Cartesian communicator has a constructor that
  takes a \indexmplshow{dimensions} object as input.
  \cxxverbatimsnippet{mplcartcreate}
\end{mplnote}

For a given communicator, you can test what type it is
with \indexmpiref{MPI_Topo_test}:
%
\cverbatimsnippet{carttypetest}

For a Cartesian communicator, you can retrieve
its information with \indexmpidef{MPI_Cartdim_get}
and \indexmpidef{MPI_Cart_get}:
%
\cverbatimsnippet{cartget}

\begin{mplnote}{Get the dimensions object}
  The \indexmplshow{dimensions} object can be extracted
  from the communicator
  \cxxverbatimsnippet{mplcartgetdims}
  after which dimensions and periodicies can be extracted:
  \cxxverbatimsnippet{mpldimsgetdims}
\end{mplnote}

\Level 1 {Cartesian vs world rank}

Each point in a Cartesian communicator has a coordinate and a rank.
The translation from rank to Cartesian coordinate is done by 
\indexmpiref{MPI_Cart_coords},
and translation from coordinates to a rank is done by
\indexmpiref{MPI_Cart_rank}.
In both cases, this translation can be done on any process;
for the latter routine note that coordinates outside the
Cartesian grid are erroneous,
if the grid is not periodic in the offending coordinate.

\cverbatimsnippet[examples/mpi/c/cart.c]{cart}

The \n{reorder} parameter to \indexmpishow{MPI_Cart_create}
indicates whether processes can have a rank
in the new communicator that is different from in the old one.

\begin{mplnote}{Rank to coord translation}
  The \lstinline{coordinates} method of the cartesian communicator
  returns a vector-like object describing the process coordinate:
  \cxxverbatimsnippet{mplcartrankcoord}
\end{mplnote}

\Level 1 {Cartesian communication}

A common communication pattern in Cartesian grids is to
do an \indexmpishow{MPI_Sendrecv} with processes that are adjacent
along one coordinate axis.

By way of example, consider a 3D grid that is periodic in the first dimension:
%
\cverbatimsnippet[examples/mpi/c/cartcoord.c]{cartperiod0}

We shift process~0 in dimensions 0 and~1.
In dimension~0 we get a wrapped-around source,
and a target that is the next process in row-major ordering;
in dimension~1 we get \indexmpishow{MPI_PROC_NULL} as source,
and a legitimate target.
%
\csnippetwithoutput{cartshift01}{examples/mpi/c}{cartshift}

\begin{mplnote}{Cartesian shifting}
  The routine \lstinline{cartesian_communicator::}\indexmplshow{shift}
  takes a dimension and a direction,
  and gives the source and destination
  as a \indexmplshow{shifted_ranks} object, which is basically
  a tuple of two integers:
  \cxxverbatimsnippet{mplcartshift}
\end{mplnote}

\Level 1 {Communicators in subgrids}

The routine \indexmpiref{MPI_Cart_sub} is similar to
\indexmpishow{MPI_Comm_split}, in that it splits a communicator
into disjoint subcommunicators.
In this case, it splits a Cartesian communicator
into disjoint Cartesian communicators,
each corresponding to a subset of the dimensions.
This subset inherits both sizes and periodicity from the original communicator.

\csnippetwithoutput{hyperplane13p}{examples/mpi/c}{cartsub}

\Level 1 {Reordering}
\label{sec:mpi-cart-map}

The \indexmpishow{MPI_Cart_create} routine has a possibility
of reordering ranks. If this is applied,
the routine \indexmpidef{MPI_Cart_map} gives the result of this.
Given the same parameters as \indexmpishow{MPI_Cart_create},
it returns the re-ordered rank for the calling process.

\Level 0 {Distributed graph topology}
\label{sec:mpi-dist-graph}

\begin{figure}[ht]
  \includegraphics[scale=.5]{graphcollective}
  \caption{Illustration of a distributed graph topology where each
    node has four neighbors}
  \label{fig:graphcollective}
\end{figure}

In many calculations on a grid (using the term in its mathematical,
\ac{FEM}, sense), a grid point will collect information from grid
points around it. Under a sensible distribution of the grid over
processes, this means that each process will collect information from
a number of neighbor processes. The number of 
neighbors is dependent on that process. For instance, in a 2D
grid (and assuming a five-point stencil for the computation) most
processes communicate with four neighbors; processes on the edge with
three, and processes in the corners with two.

Such a topology is illustrated in figure~\ref{fig:graphcollective}.

MPI's notion of \indextermbusdef{graph}{topology}, and the
\indextermsubdef{neighborhood}{collectives}, offer an elegant way of
expressing such communication structures. There are various reasons
for using graph topologies over the older, simpler methods.
\begin{itemize}
\item MPI is allowed to reorder the processes, so that network proximity
  in the cluster corresponds to proximity in the structure of the
  code.
\item Ordinary collectives could not directly be used for graph
  problems, unless one would adopt a subcommunicator for each graph
  neighborhood. However, scheduling would then lead to deadlock or
  serialization.
\item The normal way of dealing with graph problems is through
  nonblocking communications. However, since the user indicates an
  explicit order in which they are posted, congestion at certain
  processes may occur.
\item Collectives can pipeline data, while send/receive operations
  need to transfer their data in its entirety.
\item Collectives can use spanning trees, while send/receive uses a
  direct connection.
\end{itemize}

Thus the minimal description of a process graph contains for each process:
\begin{itemize}
\item Degree: the number of neighbor processes; and
\item the ranks of the processes to communicate with.
\end{itemize}
However, this ignores that communication is not always symmetric:
maybe the processes you receive from are not the ones you send
to.
Worse, maybe only one side of this duality is easily described.
Therefore, there are two routines:
\begin{itemize}
\item \indexmpishow{MPI_Dist_graph_create_adjacent} assumes that a
  process knows both who it is sending it, and who will send to
  it. This is the most work for the programmer to specify, but it is
  ultimately the most efficient.
\item \indexmpishow{MPI_Dist_graph_create} specifies on each process
  only what it is the source for; that is, who this process will be sending
  to. Consequently, some amount of processing
  --~including communication~-- is needed to build the converse
  information, the ranks that will be sending to a process.
\end{itemize}

\Level 1 {Graph creation}

There are two creation routines for process graphs. These routines are
fairly general in that they allow any process to specify any part of
the topology. In practice, of course, you will mostly let each process
describe its own neighbor structure.

The routine \indexmpishow{MPI_Dist_graph_create_adjacent} assumes that a process
knows both who it is sending it, and who will send to it. This means
that every edge in the communication graph is represented twice, so
the memory footprint is double of what is strictly necessary. However,
no communication is needed to build the graph.

The second creation routine, \indexmpiref{MPI_Dist_graph_create}, is
probably easier to use, especially in cases where the communication
structure of your program is symmetric, meaning that a process sends
to the same neighbors that it receives from.  Now you specify on each
process only what it is the source for; that is, who this process will
be sending to.\footnote{I disagree with this design
  decision. Specifying your sources is usually easier than specifying
  your destinations.}. Consequently, some amount of processing
--~including communication~-- is needed to build the converse
information, the ranks that will be sending to a process.

\begin{mplnote}{Distributed graph creation}
  The class \lstinline+mpl::+\indexmpldef{dist_graph_communicator}
  only has a constructor corresponding to \indexmpishow{MPI_Dist_graph_create}.
\end{mplnote}

Figure~\ref{fig:graphcollective} describes the common five-point
stencil structure. If we let each process only describe itself, we get
the following:
\begin{itemize}
\item \n{nsources}$=1$ because the calling process describes on node
  in the graph: itself.
\item \n{sources} is an array of length~1, containing the rank of the
  calling process.
\item \n{degrees} is an array of length~1, containing the degree
  (probably:~4) of this process.
\item \n{destinations} is an array of length the degree of this
  process, probably again~4. The elements of this array are the ranks
  of the neighbor nodes; strictly speaking the ones that this process
  will send to.
\item \n{weights} is an array declaring the relative importance of the
  destinations. For an \indextermsub{unweighted}{graph} use
  \indexmpishow{MPI_UNWEIGHTED}.
  In the case the graph is weighted, but the degree of a source is zero, you can pass
  an empty array as \indexmpidef{MPI_WEIGHTS_EMPTY}.
\item \n{reorder} (\n{int} in~C, \n{LOGICAL} in~Fortran) indicates
  whether MPI is allowed to shuffle processes to achieve greater locality.
\end{itemize}

The resulting communicator has all the processes of the original
communicator, with the same ranks.
In other words \indexmpishow{MPI_Comm_size} and \indexmpishow{MPI_Comm_rank}
gives the same values on the graph communicator, as on the intra-communicator
that it is constructed from.
To get information about the grouping,
use \indexmpishow{MPI_Dist_graph_neighbors}
and \indexmpishow{MPI_Dist_graph_neighbors_count};
section~\ref{sec:graph-neighbors}.

By way of example we build an unsymmetric graph, that is,
an edge $v_1\rightarrow v_2$ between vertices~$v_1,v_2$
does not imply an edge~$v_2\rightarrow v_1$.
%
\csnippetwithblock{distgraphsetup}{\vbox to 0pt {\includegraphics[scale=.75]{distgraph}\vss}}

Here we gather the coordinates of the source neighbors:

{
  \def\snippetcodefraction{.45}
  \def\snippetlistfraction{.5}
  \csnippetwithoutput{distgraphreadout}{examples/mpi/c}{graph}
}

However, we can't rely on the sources being ordered,
so the following segment performs an explicit query for the
source neighbors:

{
  \def\snippetcodefraction{.45}
  \def\snippetlistfraction{.5}
  \csnippetwithoutput{distgraphcount}{examples/mpi/c}{graphcount}
}

\begin{pythonnote}{Graph communicators}
  Graph communicator creation is a method of the \plstinline{Comm} class,
  and the graph communicator is a function return result:
\begin{lstlisting}
graph_comm = oldcomm.Create_dist_graph(sources, degrees, destinations)
\end{lstlisting}
  The weights, info, and reorder arguments have default values.
\end{pythonnote}

\begin{mplnote}{Graph communicators}
  \label{mpl:graph-comm}
  The constructor \indexmplshow{dist_graph_communicator}
\begin{lstlisting}
dist_graph_communicator
   (const communicator &old_comm, const source_set &ss,
    const dest_set &ds, bool reorder = true);    
\end{lstlisting}
is a wrapper around \indexmpishow{MPI_Dist_graph_create_adjacent}.
\end{mplnote}
\begin{mplnote}{Graph communicator querying}
  \label{mpl:graph-query}
  Methods \indexmplshow{indegree}, \indexmplshow{outdegree}
  are wrappers around \indexmpishow{MPI_Dist_graph_neighbors_count}.
  Sources and targets can be queried with
  \indexmplshow{inneighbors} and \indexmplshow{outneighbors},
  which are wrappers around \indexmpishow{MPI_Dist_graph_neighbors}.
\end{mplnote}

\Level 1 {Neighbor collectives}
\label{sec:mpi-neigh-coll}

We can now use the graph topology to perform a gather or allgather
\indexmpiref{MPI_Neighbor_allgather}
that combines only the processes directly connected to the calling
process.

The neighbor collectives have the same argument list as the regular
collectives, but they apply to a graph communicator.

\begin{figure}[ht]
  \includegraphics[scale=.5]{rightgraph}
  \caption{Solving the right-send exercise with neighborhood
    collectives}
  \label{fig:rightgraph}
\end{figure}

\begin{exercise}
  \label{ex:rightgraph}
  Revisit exercise~\ref{ex:serialsend} and solve it using
  \indexmpishow{MPI_Dist_graph_create}.
  Use figure~\ref{fig:rightgraph} for inspiration.

  Use a degree value of~1.
  \skeleton{rightgraph}
\end{exercise}

The previous exercise can be done with a degree value of:
\begin{itemize}
\item 1, reflecting that each process communicates with just 1 other; or
\item 2, reflecting that you really gather from two processes.
\end{itemize}
In the latter case, results do not wind up in the receive buffer
in order of increasing process number as with a traditional gather.
Rather, you need to use~\indexmpishow{MPI_Dist_graph_neighbors}
to find their sequencing; see section~\ref{sec:graph-neighbors}.

Another neighbor collective is \indexmpishow{MPI_Neighbor_alltoall}.

The vector variants are
\indexmpishow{MPI_Neighbor_allgatherv}
and
\indexmpishow{MPI_Neighbor_alltoallv}.

There is a heterogenous (multiple datatypes) variant:
\indexmpishow{MPI_Neighbor_alltoallw}.

The list is: \indexmpishow{MPI_Neighbor_allgather},
\indexmpishow{MPI_Neighbor_allgatherv},
\indexmpishow{MPI_Neighbor_alltoall},
\indexmpishow{MPI_Neighbor_alltoallv},
\indexmpishow{MPI_Neighbor_alltoallw}.

Nonblocking:
\indexmpishow{MPI_Ineighbor_allgather},
\indexmpishow{MPI_Ineighbor_allgatherv},
\indexmpishow{MPI_Ineighbor_alltoall},
\indexmpishow{MPI_Ineighbor_alltoallv},
\indexmpishow{MPI_Ineighbor_alltoallw}.

For unclear reasons there is no \indexmpishow{MPI_Neighbor_allreduce}.

\Level 1 {Query}
\label{sec:graph-neighbors}

There are two routines for querying the neighbors of a process:
%
\indexmpiref{MPI_Dist_graph_neighbors_count}
and
\indexmpiref{MPI_Dist_graph_neighbors}.

While this information seems derivable from the graph construction,
that is not entirely true for two reasons.
\begin{enumerate}
\item With the nonadjoint version \indexmpishow{MPI_Dist_graph_create},
  only outdegrees and destinations are specified; this call then supplies
  the indegrees and sources;
\item As observed above, the order in which data is placed in the
  receive buffer of a gather call is not determined by the create call,
  but can only be queried this way.
\end{enumerate}

\Level 1 {Graph topology (deprecated)}
\label{sec:mpi-1-graph}

The original \mpistandard{1} had a graph topology interface
\indexmpidef{MPI_Graph_create}
which required each process to specify the full process graph. Since
this is not scalable, it should be considered deprecated. Use the
distributed graph topology (section~\ref{sec:mpi-dist-graph}) instead.

Other legacy routines: 
\indexmpidef{MPI_Graph_neighbors},
\indexmpidef{MPI_Graph_neighbors_count},
\indexmpidef{MPI_Graph_get},
\indexmpidef{MPI_Graphdims_get}.

\Level 1 {Re-ordering}

Similar to the \indexmpishow{MPI_Cart_map} routine
(section~\ref{sec:mpi-cart-map}),
the routine \indexmpidef{MPI_Graph_map} gives a re-ordered rank
for the calling process.

