% -*- latex -*-
%%%%%%%%%%%%%%%%%%%%%%%%%%%%%%%%%%%%%%%%%%%%%%%%%%%%%%%%%%%%%%%%
%%%%%%%%%%%%%%%%%%%%%%%%%%%%%%%%%%%%%%%%%%%%%%%%%%%%%%%%%%%%%%%%
%%%%
%%%% This text file is part of the source of
%%%% `Parallel Programming in MPI and OpenMP'
%%%% by Victor Eijkhout, copyright 2012-2023
%%%%
%%%% mpi-commbasic.tex : communicator basics
%%%%
%%%%%%%%%%%%%%%%%%%%%%%%%%%%%%%%%%%%%%%%%%%%%%%%%%%%%%%%%%%%%%%%
%%%%%%%%%%%%%%%%%%%%%%%%%%%%%%%%%%%%%%%%%%%%%%%%%%%%%%%%%%%%%%%%

With \indexmpiref{MPI_Comm_dup}
you can make an exact duplicate of a communicator
(see section~\ref{sec:mpi-comm-dup-lib} for an application).
There is a nonblocking variant \indexmpiref{MPI_Comm_idup}.

These calls do not propagate info hints
(sections \ref{sec:mpi-info} and~\ref{sec:mpi-comm-info});
to achieve this,
use \indexmpidef{MPI_Comm_dup_with_info} and \indexmpidef{MPI_Comm_idup_with_info};
section~\ref{sec:mpi-comm-info}.

\begin{pythonnote}{Communicator duplication}
  Duplicate communicators are created as output of the duplication routine:
  \begin{lstlisting}
    newcomm = comm.Dup()
  \end{lstlisting}
\end{pythonnote}

\begin{mplnote}{Communicator duplication}
  Communicators can be duplicated but only during initialization.
  Copy assignment has been deleted. Thus:
\begin{lstlisting}
// LEGAL: 
mpl::communicator init = comm;
// WRONG:
mpl::communicator init;
init = comm;
\end{lstlisting}
\end{mplnote}

\Level 1 {Communicator comparing}
\label{sec:comm-compare}

You may wonder what `an exact copy' means precisely.
For this, think of a communicator as a context label that you can attach to,
among others, operations such as sends and receives.
And it's that label that counts, not what processes are in the communicator.
A~send and a receive `belong together' if they have the same communicator context.
Conversely, a~send in one communicator can not be matched
to a receive in a duplicate communicator, made by \indexmpishow{MPI_Comm_dup}.

Testing whether two communicators are really the same
is then more than testing if they comprise the same processes.
The call \indexmpidef{MPI_Comm_compare} returns \indexmpishow{MPI_IDENT}
if two communicator values are the same,
and not if one is derived from the other by duplication:
%
\csnippetwithoutput{commcopycompare}{examples/mpi/c}{commcompare}

Communicators that are not actually the same can be
\begin{itemize}
\item consisting of the same processes, in the same order,
  giving \indexmpidef{MPI_CONGRUENT};
\item merely consisting of the same processes, but not in the same order,
  giving \indexmpidef{MPI_SIMILAR};
\item different, giving \indexmpidef{MPI_UNEQUAL}.
\end{itemize}

Comparing against \indexmpishow{MPI_COMM_NULL} is not allowed.

\begin{mplnote}{Communicator comparing}
  \csnippetwithoutput{mplcommraw}{examples/mpi/mpl}{rawcompare}
\end{mplnote}

\Level 1 {Communicator duplication for library use}
\label{sec:mpi-comm-dup-lib}

Duplicating a communicator may seem pointless, but it is actually very useful for the design of
software libraries. Imagine that you have a code
\lstset{style=reviewcode,language=C}
\begin{lstlisting}
MPI_Isend(...); MPI_Irecv(...);
// library call
MPI_Waitall(...);
\end{lstlisting}
and suppose that the library has receive calls. Now it is possible that the 
receive in the library inadvertently
catches the message that was sent in the outer environment.

Let us consider an example.
First of all, here is code where the library stores the communicator
of the calling program:
%
\cxxverbatimsnippet[examples/mpi/c/commdupwrong.cxx]{wrongcatchlib}

This models a main program that does a simple message exchange, and it
makes two calls to library routines. Unbeknown to the user, the
library also issues send and receive calls, and they turn out to
interfere.

Here
\begin{itemize}
\item The main program does a send,
\item the library call \n{function_start} does a send and a receive;
  because the receive can match either send, it is paired with the
  first one;
\item the main program does a receive, which will be paired with the send of the 
  library call;
\item both the main program and the library do a wait call, and in
  both cases all requests are succesfully fulfilled, just not the way
  you intended.
\end{itemize}

To prevent this confusion, the library should duplicate the outer
communicator with \indexmpishow{MPI_Comm_dup}
%
and send all messages with respect to its duplicate. Now messages from the user
code can never reach the library software, since they are on different communicators.

\cverbatimsnippet[examples/mpi/c/commdupright.cxx]{rightcatchlib}

Note how the preceding example
performs the \indexmpishow{MPI_Comm_free}
cal in a C++ \indexterm{destructor}.

\pverbatimsnippet[examples/mpi/p/commdup.py]{catchlibp}

