% -*- latex -*-
%%%%%%%%%%%%%%%%%%%%%%%%%%%%%%%%%%%%%%%%%%%%%%%%%%%%%%%%%%%%%%%%
%%%%%%%%%%%%%%%%%%%%%%%%%%%%%%%%%%%%%%%%%%%%%%%%%%%%%%%%%%%%%%%%
%%%%
%%%% This text file is part of the source of 
%%%% `Parallel Programming in MPI and OpenMP'
%%%% by Victor Eijkhout, copyright 2012-2022
%%%%
%%%% mpi-shared.tex : about shared memory in MPI
%%%%
%%%%%%%%%%%%%%%%%%%%%%%%%%%%%%%%%%%%%%%%%%%%%%%%%%%%%%%%%%%%%%%%
%%%%%%%%%%%%%%%%%%%%%%%%%%%%%%%%%%%%%%%%%%%%%%%%%%%%%%%%%%%%%%%%

\index{shared memory|see{memory, shared}}

Some programmers are under the impression that MPI would not be efficient on
shared memory, since all operations are done through what looks like network calls.
This is not correct: many MPI
implementations have optimizations that detect shared memory and can
exploit it, so that data is copied, rather than going through a communication layer.
(Conversely, programming systems for shared memory such as \indexterm{OpenMP}
can actually have inefficiencies associated with thread handling.)
The main inefficiency associated with using MPI on shared memory is then
that processes can not actually share data.

The one-sided MPI calls (chapter~\ref{ch:mpi-onesided}) can also be used to
emulate shared memory, in the sense that an origin process can access data
from a target process without the target's active involvement.
However, these calls do not distinguish between actually shared memory
and one-sided access across the network.

In this chapter we will look at the ways MPI
can interact with the presence of actual shared memory. 
(This functionality was added in the \mpistandard{3} standard.)
This relies on the \indexmpishow{MPI_Win} windows concept, but
otherwise uses direct access of other processes' memory.

\Level 0 {Recognizing shared memory}
\label{mpi-comm-split-type}

MPI's one-sided routines take a very symmetric view of processes:
each process can access the window of every other process (within a communicator).
Of course, in practice there will be a difference in performance
depending on whether the origin and target are actually
on the same shared memory, or whether they can only communicate through the network.
For this reason MPI makes it easy to group processes by shared memory domains
using \indexmpiref{MPI_Comm_split_type}.

Here the \n{split_type} parameter has to be from the following (short) list:
\begin{itemize}
\item \indexmpishow{MPI_COMM_TYPE_SHARED}: split the communicator into subcommunicators
  of processes sharing a memory area.
\begin{mpifournote}{Split by guided types}
  \item \indexmpishow{MPI_COMM_TYPE_HW_GUIDED} (\mpistandard{4}):
    split using an \n{info} value from \indexmpidef{MPI_Get_hw_resource_types}.
    % https://github.com/mpi-forum/mpi-issues/issues/132
  \item \indexmpishow{MPI_COMM_TYPE_HW_UNGUIDED} (\mpistandard{4}):
    similar to \indexmpishow{MPI_COMM_TYPE_HW_GUIDED}, but the resulting communicators
    should be a strict subset of the original communicator.
    On processes where this condition can not be fullfilled,
    \indexmpishow{MPI_COMM_NULL} will be returned.
\end{mpifournote}
\end{itemize}

Splitting by shared memory:
%
\begingroup
\def\snippetcodefraction{.45}
\def\snippettextfraction{.5}
\csnippetwithoutput{commsplittype}{examples/mpi/c}{commsplittype}
\endgroup

\begin{exercise}
  \label{ex:mpi-node-proc-split}
  Write a program that uses \indexmpishow{MPI_Comm_split_type}
  to  analyze for a run
  \begin{enumerate}
  \item How many nodes there are;
  \item How many processes there are on each node.
  \end{enumerate}
  If you run this program on an unequal distribution,
  say 10~processes on 3~nodes, what distribution do you find?
  \lstinputlisting{examples/mpi/c/nodecount.runout}
\end{exercise}

\begin{remark}
  The \indexterm{OpenMPI} implementation of MPI has a number of
  non-standard split types, such as \indexmpishow{OMPI_COMM_TYPE_SOCKET};
  see \url{https://www.open-mpi.org/doc/v4.1/man3/MPI_Comm_split_type.3.php}
\end{remark}

\begin{mplnote}{Split by shared memory}
  Similar to ordinary communicator splitting~\ref{mpl::split}:
  \lstinline+communicator::+\indexmpldef{split_shared}.
\end{mplnote}

\Level 0 {Shared memory for windows}

Processes that exist on the same physical shared memory should be able
to move data by copying, rather than through MPI send/receive calls
--~which of course will do a copy operation under the hood.
In order to do such user-level copying:
\begin{enumerate}
\item We need to create a shared memory area with
  \indexmpishow{MPI_Win_allocate_shared}.
  This creates a window with the
  \emph{unified memory model}\index{window!memory!model!unified}
  (see section~\ref{sec:win-model});
  and
\item We need to get pointers to where a process' area is in this
  shared space; this is done with \indexmpishow{MPI_Win_shared_query}.
\end{enumerate}

\Level 1 {Pointers to a shared window}

The first step is to create a window (in the sense of one-sided MPI;
section~\ref{sec:windows}) on the processes on one node.
Using the \indexmpiref{MPI_Win_allocate_shared} call presumably will
put the memory close to the 
\indexterm{socket} on which the process runs.

\cverbatimsnippet[examples/mpi/c/sharedbulk.c]{mpisharedwindow}

The memory allocated by \indexmpishow{MPI_Win_allocate_shared} is
contiguous between the processes. This makes it possible to do address
calculation. However, if a cluster node has a \ac{NUMA} structure, for
instance if two sockets have memory directly attached to each, this
would increase latency for some processes. To prevent this, the key
\indexmpishow{alloc_shared_noncontig} can be set to \n{true} in the
\indexmpishow{MPI_Info} object.
\begin{mpifournote}{Window memory alignment}
  In the contiguous case, the \indexmpishow{mpi_minimum_memory_alignment}
  info argument
  (section~\ref{sec:win-alloc})
  applies only to the memory on the first process;
  in the noncontiguous case it applies to all.
\end{mpifournote}

\cverbatimsnippet[examples/mpi/c/numa.c]{winnoncontig}

Let's explore this.
We create a shared window where each process stores exactly one double,
that is, 8~bytes.
The following code fragment queries the window locations,
and prints the distance in bytes to the window on process~0.
%
\cverbatimsnippet[examples/mpi/c/numa.c]{wincontigquery}

With the default strategy, these windows are contiguous,
and so the distances are multiples of~8 bytes.
Not so for the the non-contiguous allocation:
%
\begin{multicols}{2}
Strategy: default behavior of shared window allocation
\begin{verbatim}
Distance 1 to zero: 8
Distance 2 to zero: 16
Distance 3 to zero: 24
Distance 4 to zero: 32
Distance 5 to zero: 40
Distance 6 to zero: 48
Distance 7 to zero: 56
Distance 8 to zero: 64
Distance 9 to zero: 72
\end{verbatim}
\columnbreak
Strategy: allow non-contiguous shared window allocation
\begin{verbatim}
Distance 1 to zero: 4096
Distance 2 to zero: 8192
Distance 3 to zero: 12288
Distance 4 to zero: 16384
Distance 5 to zero: 20480
Distance 6 to zero: 24576
Distance 7 to zero: 28672
Distance 8 to zero: 32768
Distance 9 to zero: 36864
\end{verbatim}
\end{multicols}

The explanation here is that each window is placed 
on its own \indextermsub{small}{page},
which on this particular system has a size of~4K.

\begin{remark}
  The ampersand operator in~C is not a
  \indextermsub{physical}{address}, but a
  \indextermsub{virtual}{address}.
  The translation of where pages are placed in physical memory
  is determined by the \indextermbus{page}{table}.
\end{remark}

\Level 1 {Querying the shared structure}

Even though the window created above is shared, that doesn't mean it's
contiguous. Hence it is necessary to retrieve the pointer to the area
of each process that you want to communicate with:
\indexmpiref{MPI_Win_shared_query}.

\cverbatimsnippet[examples/mpi/c/sharedbulk.c]{mpisharedpointer}

\Level 1 {Heat equation example}

As an example, which consider the 1D heat equation. On each process we
create a local area of three point:
%
\cverbatimsnippet[examples/mpi/c/sharedshared.c]{allocateshared3pt}

\Level 1 {Shared bulk data}

In applications such as \indexterm{ray tracing}, there is a read-only
large data object (the objects in the scene to be rendered) that is
needed by all processes. In traditional MPI, this would need to be
stored redundantly on each process, which leads to large memory
demands. With MPI shared memory we can store the data object once per
node. Using as above \indexmpishow{MPI_Comm_split_type} to find a
communicator per \ac{NUMA} domain, we store the object on process zero
of this node communicator.

\begin{exercise}
  \label{ex:shareddata}
  Let the `shared' data originate on process zero in
  \indexmpishow{MPI_COMM_WORLD}. Then:
  \begin{itemize}
  \item create a communicator per shared memory domain;
  \item create a communicator for all the processes with number zero on their
    node;
  \item broadcast the shared data to the processes zero on each node.
  \end{itemize}
  \skeleton{shareddata}
\end{exercise}
