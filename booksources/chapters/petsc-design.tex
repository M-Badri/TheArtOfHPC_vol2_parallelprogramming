% -*- latex -*-
%%%%%%%%%%%%%%%%%%%%%%%%%%%%%%%%%%%%%%%%%%%%%%%%%%%%%%%%%%%%%%%%
%%%%%%%%%%%%%%%%%%%%%%%%%%%%%%%%%%%%%%%%%%%%%%%%%%%%%%%%%%%%%%%%
%%%%
%%%% This text file is part of the source of
%%%% `Parallel Programming in MPI and OpenMP'
%%%% by Victor Eijkhout, copyright 2012-2022
%%%%
%%%% petsc-design.tex : a tutorial section
%%%%
%%%%%%%%%%%%%%%%%%%%%%%%%%%%%%%%%%%%%%%%%%%%%%%%%%%%%%%%%%%%%%%%
%%%%%%%%%%%%%%%%%%%%%%%%%%%%%%%%%%%%%%%%%%%%%%%%%%%%%%%%%%%%%%%%

\Level 0 {What is PETSc and why?}

PETSc is a library with a great many uses, but for now let's say that
it's primarily a library for dealing with the sort of linear algebra
that comes from discretized \acp{PDE}. On a single processor, the
basics of such computations 
can be coded out by a grad student during a semester
course in numerical analysis, but on large scale issues get much more
complicated and a library becomes indispensible.

PETSc's prime justification is then that it helps you realize
scientific computations at large scales, meaning large problem sizes
on large numbers of processors.

There are two points to emphasize here:
\begin{itemize}
\item Linear algebra with dense matrices is relatively simple to
  formulate. For sparse matrices the amount of logistics in dealing
  with nonzero patterns increases greatly. PETSc does most of that for
  you.
\item Linear algebra on a single processor, even a multicore one, is
  managable; distributed memory parallelism is much harder, and
  distributed memory sparse linear algebra operations are doubly
  so. Using PETSc will save you many, many, Many! hours of coding over
  developing everything yourself from scratch.
\end{itemize}

\begin{remark}
  The PETSc library has hundreds of routines. In this chapter and the
  next few we will only touch on a basic subset of these. The full
  list of man pages can be found at
  \url{https://petsc.org/release/docs/manualpages/singleindex.html}.
  Each man page comes with links to related routines, as well as (usually)
  example codes for that routine.
\end{remark}

\Level 1 {What is in PETSc?}

The routines in PETSc (of which there are hundreds) can roughly be
divided in these classes:
\begin{itemize}
\item Basic linear algebra tools: dense and sparse matrices, both
  sequential and parallel, their construction and simple operations.
\item Solvers for linear systems, and to a lesser extent nonlinear
  systems; also time-stepping methods.
\item Profiling and tracing: after a successful run, timing for
  various routines can be given. In case of failure, there are
  traceback and memory tracing facilities.
\end{itemize}

\Level 1 {Programming model}

PETSc, being based on MPI, uses the \ac{SPMD} programming model
(section~\ref{sec:spmd}), where all processes execute the same
executable. Even more than in regular MPI codes, this makes sense
here, since most PETSc objects are collectively created on some
communicator, often \indexmpishow{MPI_COMM_WORLD}. With the
object-oriented design (section~\ref{sec:petsc-oop}) this means that a
PETSc program almost looks like a sequential program.
\begin{lstlisting}
MatMult(A,x,y);      // y <- Ax
VecCopy(y,res);      // r <- y
VecAXPY(res,-1.,b);  // r <- r - b
\end{lstlisting}
This is
sometimes called \indextermbus{sequential}{semantics}.

\Level 1 {Design philosophy}
\label{sec:petsc-oop}

PETSc has an object-oriented design, even though it is written
in~C. There are classes of objects, such as \clstinline{Mat} for
matrices and \clstinline{Vec} for Vectors, but there is also the
\clstinline{KSP} (for "Krylov SPace solver") class of linear system solvers, and
\clstinline{PetscViewer} for outputting matrices and vectors to screen or file.

Part of the object-oriented design is the polymorphism of objects:
after you have created a \clstinline{Mat} matrix as sparse or dense, all methods
such as MatMult (for the matrix-vector product) take the same
arguments: the matrix, and an input and output vector.

This design where the programmer manipulates a `handle' also means
that the internal of the object, the actual storage of the elements,
is hidden from the programmer. This hiding goes so far that even
filling in elements is not done directly but through function calls:
\begin{lstlisting}
VecSetValue(i,j,v,mode)
MatSetValue(i,j,v,mode)
MatSetValues(ni,is,nj,js,v,mode)
\end{lstlisting}

\Level 1 {Language support}

\Level 2 {C/C++}

PETSc is implemented in C, so there is a natural interface
to~C. There is no separate C++ interface.

\Level 2 {Fortran}

A~\emph{Fortran90}\index{Fortran!Fortran90!PETSc interface}
interface exists. The \emph{Fortran77}\index{Fortran!Fortran77!PETSc interface}
interface is only of
interest for historical reasons.

To use Fortran, include both a module and a cpp header file:
\begin{verbatim}
#include "petsc/finclude/petscXXX.h"
use petscXXX
\end{verbatim}
(here \n{XXX} stands for one of the PETSc types, but including
\flstinline{petsc.h} and using \flstinline{use petsc}
gives inclusion of the whole library.)

Variables can be declared with their type (\clstinline{Vec},
\clstinline{Mat}, \clstinline{KSP} et cetera), but internally they are
Fortran \clstinline{Type} objects so they can be declared as such.

Example:
\begin{lstlisting}
#include "petsc/finclude/petscvec.h"
use petscvec
Vec b
type(tVec) x
\end{lstlisting}

The output arguments of many query routines are optional in PETSc.
While in C a generic \lstinline{NULL} can be passed,
Fortran has type-specific nulls, such as
\indexpetsctt{PETSC_NULL_INTEGER}, \indexpetsctt{PETSC_NULL_OBJECT}.

\Level 2 {Python}
\label{sec:py-interface}

A \emph{python}\index{Python!PETSc interface} interface was written by
\indextermsub{Lisandro}{Dalcin}.
It can be added to to PETSc at installation time;
section~\ref{sec:petsc-install}.

This book discusses the Python interface
in short remarks in the appropriate sections.

\Level 1 {Documentation}

PETSc comes with a manual in pdf form and web pages with the
documentation for every routine. The starting point is the web page
\url{https://petsc.org/release/documentation/}.

There is also a mailing list with excellent support for questions and
bug reports.
\begin{taccnote}
  For questions specific to using PETSc on TACC resources, submit
  tickets to the \emph{TACC}\index{TACC!portal} or
  \indextermbus{XSEDE}{portal}.
\end{taccnote}

\Level 0 {Basics of running a PETSc program}

\Level 1 {Compilation}

A PETSc compilation needs a number of include and library paths,
probably too many to specify interactively. The easiest solution is to
create a makefile and load the standard variables and compilation rules.
(You can use \n{$PETSC_DIR/share/petsc/Makefile.user} for inspiration.)

Throughout, we will assume that variables \indexpetscshow{PETSC_DIR} and
\indexpetscshow{PETSC_ARCH} have been set.
These depend on your local installation; see section~\ref{sec:petsc-install}.

In the easiest setup, you leave the compilation to PETSc and
your make rules only do the link step, using \indextermtt{CLINKER}
or \indextermtt{FLINKER} for C/Fortran respectively:
\begin{verbatim}
include ${PETSC_DIR}/lib/petsc/conf/variables
include ${PETSC_DIR}/lib/petsc/conf/rules
program : program.o
        ${CLINKER} -o $@ $^ ${PETSC_LIB}
\end{verbatim}
The two include lines provide the compilation rule and the library
variable.

You can use these rules:
\begin{verbatim}
% : %.F90
        $(LINK.F) -o $@ $^ $(LDLIBS)
%.o: %.F90
        $(COMPILE.F) $(OUTPUT_OPTION) $<
% : %.cxx
        $(LINK.cc) -o $@ $^ $(LDLIBS)
%.o: %.cxx
        $(COMPILE.cc) $(OUTPUT_OPTION) $<

## example link rule:
# app : a.o b.o c.o
#       $(LINK.F) -o $@ $^ $(LDLIBS)
\end{verbatim}
\begin{comment}
include ${PETSC_DIR}/lib/petsc/conf/variables
%.o : %.c
        ${CC} -c $^ ${PETSC_CC_INCLUDES}
program : program.o
        ${CLINKER} -o $@ $^ ${PETSC_LIB}
\end{comment}
(The \indexpetscshow{PETSC_CC_INCLUDES} variable contains all paths for
compilation of C~programs; correspondingly there is
\indexpetscshow{PETSC_FC_INCLUDES} for Fortran source.)

If don't want to include those configuration files, you can find out
the include options by:
\begin{verbatim}
cd $PETSC_DIR
make getincludedirs
make getlinklibs
\end{verbatim}
and copying the results into your compilation script.

There is an example makefile \verb+$PETSC_DIR/share/petsc/Makefile.user+
you can take for inspiration. Invoked without arguments it prints out
the relevant variables:
\begin{verbatim}
[c:246] make -f ! $PETSC_DIR/share/petsc/Makefile.user
CC=/Users/eijkhout/Installation/petsc/petsc-3.13/macx-clang-debug/bin/mpicc
CXX=/Users/eijkhout/Installation/petsc/petsc-3.13/macx-clang-debug/bin/mpicxx
FC=/Users/eijkhout/Installation/petsc/petsc-3.13/macx-clang-debug/bin/mpif90
CFLAGS=-Wall -Wwrite-strings -Wno-strict-aliasing -Wno-unknown-pragmas -fstack-protector -Qunused-arguments -fvisibility=hidden -g3
CXXFLAGS=-Wall -Wwrite-strings -Wno-strict-aliasing -Wno-unknown-pragmas -fstack-protector -fvisibility=hidden -g
FFLAGS=-m64 -g
CPPFLAGS=-I/Users/eijkhout/Installation/petsc/petsc-3.13/macx-clang-debug/include -I/Users/eijkhout/Installation/petsc/petsc-3.13/include
LDFLAGS=-L/Users/eijkhout/Installation/petsc/petsc-3.13/macx-clang-debug/lib -Wl,-rpath,/Users/eijkhout/Installation/petsc/petsc-3.13/macx-clang-debug/lib
LDLIBS=-lpetsc -lm
\end{verbatim}

\begin{taccnote}
  On TACC clusters, a petsc installation is loaded by commands such as
\begin{verbatim}
module load petsc/3.16
\end{verbatim}
Use \n{module avail petsc} to see what configurations exist. The basic
versions are
\begin{verbatim}
# development
module load petsc/3.11-debug
# production
module load petsc/3.11
\end{verbatim}
Other installations are real versus complex, or 64bit integers instead
of the default 32. The command 
\begin{verbatim}
module spider petsc
\end{verbatim}
tells you all the
available petsc versions. The listed modules have a naming convention
such as \n{petsc/3.11-i64debug} where the 3.11 is the PETSc release (minor
patches are not included in this version; TACC aims to install only
the latest patch, but generally several versions are available), and
\n{i64debug} describes the debug version of the installation with 64bit
integers.
\end{taccnote}

\Level 1 {Running}

PETSc programs use MPI for parallelism, so they are started like any other
MPI program:
\begin{verbatim}
mpiexec -n 5 -machinefile mf \
    your_petsc_program option1 option2 option3
\end{verbatim}
\begin{taccnote}
  On TACC clusters, use \indextermtt{ibrun}.
\end{taccnote}

\Level 1 {Initialization and finalization}
\label{sec:petscinit}

PETSc has an call that initializes both PETSc and MPI, so normally you
would replace \indexmpishow{MPI_Init} by
\indexpetscref{PetscInitialize}.
Unlike with MPI, you do not want to
use a \n{NULL} value for the \n{argc,argv} arguments, since PETSc
makes extensive use of commandline options; see
section~\ref{sec:petsc-options}.

\cverbatimsnippet[examples/petsc/c/init.c]{petscinit}

There are two further arguments to \indexpetscshow{PetscInitialize}:
\begin{enumerate}
\item the name of an options database file; and
\item a help string, that is displayed if you run your program with the \n{-h} option.
\end{enumerate}

\begin{fortrannote}{Petsc Initialization}
  The Fortran version has no arguments for commandline options;
  however, you can pass a file of database options:
\begin{lstlisting}
PetscInitialize(filename,ierr)    
\end{lstlisting}
  If none is specified, give \indexpetsctt{PETSC_NULL_CHARACTER} as argument.

  For passing help information there is a variant that takes a help string:
  \fsnippetwithoutput{petschelpf}{examples/petsc/f}{mainhelp}

  If your main program is in~C, but some of your PETSc calls are
    in Fortran files, it is necessary to call
    \indexpetscdef{PetscInitializeFortran} after
    \indexpetscshow{PetscInitialize}.
  \fverbatimsnippet[examples/petsc/f/init.F90]{petscinitf}
\end{fortrannote}

\begin{pythonnote}{Init, and with commandline options}
  The following works if you don't need commandline options.
\begin{verbatim}
from petsc4py import PETSc
\end{verbatim}
To pass commandline arguments to PETSc, do:
\begin{verbatim}
import sys
from petsc4py import init
init(sys.argv)
from petsc4py import PETSc
\end{verbatim}
\end{pythonnote}

After initialization, you can use \indexmpishow{MPI_COMM_WORLD} or
\indexpetscdef{PETSC_COMM_WORLD}
(which is created by \indexmpishow{MPI_Comm_dup} and used internally by PETSc):

\begin{lstlisting}
MPI_Comm comm = PETSC_COMM_WORLD;
MPI_Comm_rank(comm,&mytid);
MPI_Comm_size(comm,&ntids);
\end{lstlisting}

\begin{pythonnote}{Communicator object}
\begin{verbatim}
comm = PETSc.COMM_WORLD
nprocs = comm.getSize(self) 
procno = comm.getRank(self)
\end{verbatim}
\end{pythonnote}

The corresponding call to replace \indexmpishow{MPI_Finalize} is
\indexpetscdef{PetscFinalize}.
You can elegantly capture and return the error code by the idiom
\begin{lstlisting}
return PetscFinalize();
\end{lstlisting}
at the end of your main program.

\Level 0 {PETSc installation}
\label{sec:petsc-install}

PETSc has a large number of installation options. These can roughly be
divided into:
\begin{enumerate}
\item Options to describe the environment in which PETSc is being
  installed, such as the names of the compilers or the location of the
  MPI library;
\item Options to specify the type of PETSc installation: real versus
  complex, 32~versus 64-bit integers, et cetera;
\item Options to specify additional packages to download.
\end{enumerate}

For an existing installation, you can find the options used,
and other aspects of the build history,
in the \indextermtt{configure.log}~/ \indextermtt{make.log}
files\index{PETSc!log files}:
\begin{verbatim}
$PETSC_DIR/$PETSC_ARCH/lib/petsc/conf/configure.log
$PETSC_DIR/$PETSC_ARCH/lib/petsc/conf/make.log
\end{verbatim}

\Level 1 {Versions}

PETSc is up to version \n{3.18.x} as of this writing.
Older versions may miss certain routines, or display certain bugs.
However, older versions may also contain routines and keywords
that have subsequently been removed. PETSc version are not
backwards compatible!

The version is stored in macros
\indexpetscshow{PETSC_VERSION},
\indexpetscshow{PETSC_VERSION_MAJOR},
\indexpetscshow{PETSC_VERSION_MINOR},
\indexpetscshow{PETSC_VERSION_SUBMINOR}.

For testing, the following macros are defined:
\indexpetscshow{PETSC_VERSION_EQ/LT/LE/GT/GE}
Example:
\cverbatimsnippet{petscversionlt}

\Level 1 {Debug}

For any set of options, you will typically make two installations:
one with \n{-with-debugging=yes} and once \n{no}.
See section~\ref{sec:petsc-debug-mode} for more detail
on the differences between debug and non-debug mode.

\Level 1 {Environment options}

Compilers, compiler options, MPI.

While it is possible to specify \indexpetscoption{download_mpich},
this should only be done on machines that you are certain do not
already have an MPI library, such as your personal
laptop. Supercomputer clusters are likely to have an optimized MPI
library, and letting PETSc download its own will lead to degraded
performance.

\Level 1 {Variants}

\begin{itemize}
\item Scalars: the option \indexpetscoption{with-scalar-type} has values
  \n{real}, \n{complex}; \indexpetscoption{with-precision} has values
  \n{single}, \n{double}, \n{__float128}, \n{__fp16}.
\end{itemize}

\Level 0 {External packages}
\label{sec:petsc-external}

PETSc can extend its functionality through external packages such as
\indexterm{mumps}, \indexterm{Hypre}, \indexterm{fftw}. These can be
specified in two ways:
\begin{enumerate}
\item Referring to an installation already on your system:
\begin{verbatim}
--with-hdf5-include=${TACC_HDF5_INC}
--with-hf5_lib=${TACC_HDF5_LIB}
\end{verbatim}
\item By letting petsc download and install them itself:
\begin{verbatim}
--with-parmetis=1 --download-parmetis=1
\end{verbatim}
\end{enumerate}

\begin{pythonnote}{petsc4py interface}
  The Python interface (section~\ref{sec:py-interface})
  can be installed with the option
\begin{verbatim}
--download-petsc4py=<no,yes,filename,url>
\end{verbatim}
This is easiest if your python already includes \indextermtt{mpi4py};
see section~\ref{sec:python-bind}.
\end{pythonnote}

\begin{remark}
  There are two packages that PETSc is capable of downloading and install,
  but that you may want to avoid:
  \begin{itemize}
  \item \n{fblaslapack}: this gives you BLAS/LAPACK through the
    Fortran `reference implementation'. If you have an optimized
    version, such as Intel's \indexterm{mkl} available, this will give
    much higher performance.
  \item \n{mpich}: this installs a MPI implementation, which may be
    required for your laptop. However, supercomputer clusters will
    already have an MPI implementation that uses the high-speed
    network. PETSc's downloaded version does not do that. Again,
    finding and using the already installed software may greatly
    improve your performance.
  \end{itemize}
\end{remark}

\Level 1 {Slepc}
\label{sec:petsc-slepc}

Most external packages add functionality to the lower layers of Petsc.
For instance, the \indexterm{Hypre} package
adds some preconditioners to Petsc's repertoire
(section~\ref{sec:petsc-prec-list}),
while \indexterm{Mumps}
(section~\ref{sec:petsc-direct})
makes it possible to use the LU
preconditioner in parallel.

On the other hand, there are packages
that use Petsc as a lower level tool.
In particular, the eigenvalue solver package \indextermdef{Slepc}~\cite{slepc-homepage}
can be installed through the options
\begin{verbatim}
--download-slepc=<no,yes,filename,url>
       Download and install slepc  current: no
--download-slepc-commit=commitid
       The commit id from a git repository to use for the build of slepc  current: 0
--download-slepc-configure-arguments=string
       Additional configure arguments for the build of SLEPc
\end{verbatim}
The slepc header files wind up in the same directory as the petsc headers,
so no change to your compilation rules are needed.
However, you need to add \n{-lslepc} to the link line.
