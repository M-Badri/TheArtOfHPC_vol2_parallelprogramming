% -*- latex -*-
%%%%%%%%%%%%%%%%%%%%%%%%%%%%%%%%%%%%%%%%%%%%%%%%%%%%%%%%%%%%%%%%
%%%%%%%%%%%%%%%%%%%%%%%%%%%%%%%%%%%%%%%%%%%%%%%%%%%%%%%%%%%%%%%%
%%%%
%%%% This text file is part of the source of 
%%%% `Parallel Programming in MPI and OpenMP'
%%%% by Victor Eijkhout, copyright 2012-2022
%%%%
%%%% mpi-proc.tex : about point-to-point communication
%%%%
%%%%%%%%%%%%%%%%%%%%%%%%%%%%%%%%%%%%%%%%%%%%%%%%%%%%%%%%%%%%%%%%
%%%%%%%%%%%%%%%%%%%%%%%%%%%%%%%%%%%%%%%%%%%%%%%%%%%%%%%%%%%%%%%%

In this course we have up to now only considered the \ac{SPMD} model
of running MPI programs.  In some rare cases you may want to run in an
\ac{MPMD} mode, rather than \ac{SPMD}. This can be achieved either on
the \ac{OS} level, using options of the \indextermtt{mpiexec} mechanism,
or you can use MPI's built-in process management. Read on if you're
interested in the latter.

\Level 0 {Process spawning}
\label{sec:mpi-dynamic}

The first version of MPI did not contain any process management
routines, even though the earlier \indexterm{PVM} project did have
that functionality. Process management was later added with \mpistandard{2}.

Unlike what you might think, newly added processes do not become part
of \indexmpishow{MPI_COMM_WORLD}; rather, they get their own communicator, and an
\indextermsubh{inter}{communicator} (section~\ref{sec:mpi-intercomm})
is established between this new group
and the existing one. The first routine is
\indexmpiref{MPI_Comm_spawn}, which tries to fire up multiple copies
of a single named executable.
Errors in starting up these codes are returned in an array of integers, or
if you're feeling sure of yourself, specify \indexmpidef{MPI_ERRCODES_IGNORE}.

It is not immediately clear whether there is opportunity for spawning
new executables; after all, \indexmpishow{MPI_COMM_WORLD} contains all
your available processors. You can probably tell your job starter to
reserve space for a few extra processes, but that is
installation-dependent (see below). However, there is a standard
mechanism for querying whether such space has been reserved.  The
attribute \indexmpidef{MPI_UNIVERSE_SIZE}, retrieved with
\indexmpishow{MPI_Comm_get_attr} (section~\ref{sec:mpi_attr}), will tell
you to the total number of hosts available.

If this option is not supported, you can determine yourself how many
processes you want to spawn.
However, if you exceed the hardware resources,
your multi-tasking operating system (which is some variant of Unix for
almost everyone) will use \indexterm{time-slicing} to start the
spawned processes, but you will not gain any performance.

Here is an example of a work manager.
%
First we query how much space we have for new processes,
using the flag to see if this option is supported:
%
\cverbatimsnippet[examples/mpi/c/spawnapp.c]{spawnmanagerq}

(See section~\ref{sec:mpi_attr} for that dereference behavior.)

Then we actually spawn the processes:
%
\cverbatimsnippet[examples/mpi/c/spawnmanager.c]{spawnmanager}
%
\pverbatimsnippet[examples/mpi/p/spawnmanager.py]{spawnmanagerp}
%
A process can detect whether it was a spawning or a spawned process
by using \indexmpishow{MPI_Comm_get_parent}:
the resulting intercommunicator is \indexmpishow{MPI_COMM_NULL}
on the parent processes.
\cverbatimsnippet[examples/mpi/c/spawnapp.c]{commparentdetect}

The spawned program looks very much like a regular MPI program, with
its own initialization and finalize calls.

\cverbatimsnippet[examples/mpi/c/spawnworker.c]{spawnworker}
%
\pverbatimsnippet[examples/mpi/p/spawnworker.py]{spawnworkerp}

Spawned processes wind up with a value of \indexmpishow{MPI_COMM_WORLD} of their
own, but managers and workers can find each other regardless.
The spawn routine returns the intercommunicator to the parent; the children
can find it through \indexmpishow{MPI_Comm_get_parent} (section~\ref{sec:mpi-inter-query}).
The number of spawning processes can be found through
\indexmpishow{MPI_Comm_remote_size} on the parent communicator.

\begin{verbatim}
Running spawnapp with usize=12, wsize=4
%%
%% manager output
%%
A universe of size 12 leaves room for 8 workers
.. spawning from c209-026.frontera.tacc.utexas.edu
%%
%% worker output
%%
Worker deduces 8 workers and 4 parents
I detect I am worker 0/8 running on c209-027.frontera.tacc.utexas.edu
I detect I am worker 1/8 running on c209-027.frontera.tacc.utexas.edu
I detect I am worker 2/8 running on c209-027.frontera.tacc.utexas.edu
I detect I am worker 3/8 running on c209-027.frontera.tacc.utexas.edu
I detect I am worker 4/8 running on c209-028.frontera.tacc.utexas.edu
I detect I am worker 5/8 running on c209-028.frontera.tacc.utexas.edu
I detect I am worker 6/8 running on c209-028.frontera.tacc.utexas.edu
I detect I am worker 7/8 running on c209-028.frontera.tacc.utexas.edu
\end{verbatim}

\Level 1 {MPI startup with universe}

You could start up a single copy of this program with 
\begin{verbatim}
mpiexec -n 1 spawnmanager
\end{verbatim}
but with a hostfile that has more than one host.

\begin{taccnote}
\indextermbus{Intel}{MPI} requires you to pass an option \n{-usize} to
\indextermtt{mpiexec} indicating the size of the comm universe. With the TACC
jobs starter \indextermtt{ibrun} do the following:
\begin{verbatim}
export FI_MLX_ENABLE_SPAWN=yes
# specific
MY_MPIRUN_OPTIONS="-usize 8" ibrun -np 4 spawnmanager
# more generic
MY_MPIRUN_OPTIONS="-usize ${SLURM_NPROCS}" ibrun -np 4 spawnmanager
# using mpiexec:
mpiexec -np 2 -usize ${SLURM_NPROCS} spawnmanager
\end{verbatim}
\end{taccnote}

\Level 1 {MPMD}

Instead of spawning a single executable, you can spawn multiple with
\indexmpidef{MPI_Comm_spawn_multiple}.
In that case a process can retrieve with
the attribute
\indexmpidef{MPI_APPNUM}
which of the executables it is;
section~\ref{sec:mpi_attr}.

\Level 0 {Socket-style communications}

It is possible to establish connections with running MPI programs that
have their own world communicator.
\begin{itemize}
\item The \indexterm{server} process establishes a port with 
  \indexmpishow{MPI_Open_port}, and calls \indexmpishow{MPI_Comm_accept} to accept
  connections to its port.
\item The \indexterm{client} process specifies that port 
  in an \indexmpishow{MPI_Comm_connect} call. This establishes the connection.
\end{itemize}

\Level 1 {Server calls}

The server calls \indexmpiref{MPI_Open_port}, yielding a port name.
Port names are generated by the system and copied into a character
buffer of length at most \indexmpidef{MPI_MAX_PORT_NAME}.

The server then needs to call 
\indexmpiref{MPI_Comm_accept} prior to the client doing a connect call.
This is collective over the calling communicator.
It returns an intercommunicator (section~\ref{sec:mpi-intercomm})
that allows communication with the client.

\cxxverbatimsnippet[examples/mpi/c/portapp.c]{mpiportmanage}

The port can be closed with 
\indexmpidef{MPI_Close_port}.

\Level 1 {Client calls}

After the server has generated a port name, the client 
needs to connect to it with
\indexmpiref{MPI_Comm_connect}, again specifying the port through a character buffer.
The connect call is collective over its communicator.

\cxxverbatimsnippet[examples/mpi/c/portapp.c]{mpiportwork}

If the named port does not exist (or has been closed),
\indexmpishow{MPI_Comm_connect} raises an error of class \indexmpishow{MPI_ERR_PORT}.

The client can sever the connection with
\indexmpishow{MPI_Comm_disconnect}.

Running the above code on 5 processes gives:
\begin{small}
\begin{verbatim}
# exchange port name:
Host sent port <<tag#0$OFA#000010e1:0001cde9:0001cdee$rdma_port#1024$rdma_host#10:16:225:0:1:205:199:254:128:0:0:0:0:0:0$>>
Worker received port <<tag#0$OFA#000010e1:0001cde9:0001cdee$rdma_port#1024$rdma_host#10:16:225:0:1:205:199:254:128:0:0:0:0:0:0$>>

# Comm accept/connect
host accepted connection
4 workers connected to 1 managers

# Send/recv over the intercommunicator
Manager sent 4 items over intercomm
Worker zero received data  
\end{verbatim}
\end{small}

\Level 1 {Published service names}
\label{sec:mpi-publish}

More elegantly than the port mechanism above,
it is possible to publish a named service, 
with \indexmpiref{MPI_Publish_name},
which can then be discovered by other processes.

\cverbatimsnippet[examples/mpi/c/publishapp.c]{publishmanager}

Worker processes connect to the intercommunicator by

\cverbatimsnippet[examples/mpi/c/publishapp.c]{publishworker}

For this it is necessary to have a \indexterm{name server} running.

\begin{intelnote}
Start the \indexterm{hydra} name server and use the corresponding mpi starter:
\begin{verbatim}
hydra_nameserver &
MPIEXEC=mpiexec.hydra
\end{verbatim}
There is an environment variable, but that doesn't seem to be needed.
\begin{verbatim}
export I_MPI_HYDRA_NAMESERVER=`hostname`:8008
\end{verbatim}
It is also possible to specify the name server as an argument to the job starter.
\end{intelnote}

At the end of a run, the service should be unpublished with
\indexmpiref{MPI_Unpublish_name}.
Unpublishing a nonexisting or already unpublished service gives an
error code of \indexmpidef{MPI_ERR_SERVICE}.

MPI provides no guarantee of fairness in servicing connection
attempts. That is, connection attempts are not necessarily satisfied
in the order in which they were initiated, and competition from other
connection attempts may prevent a particular connection attempt from
being satisfied.

\Level 1 {Unix sockets}

It is also possible to create an
\emph{intercommunicator} from a Unix
\indexterm{socket}\index{communicator!inter!from socket}
with
\indexmpiref{MPI_Comm_join}.

\Level 0 {Sessions}
\label{sec:session}

The most common way of initializing MPI,
with \indexmpishow{MPI_Init} (or \indexmpishow{MPI_Init_thread}) and \indexmpishow{MPI_Finalize},
is known as the \indextermdef{world model} which can be described as:
\begin{enumerate}
\item There is a single call to \indexmpishow{MPI_Init} or \indexmpishow{MPI_Init_thread};
\item There is a single call to \indexmpishow{MPI_Finalize};
\item With very few exceptions, all MPI calls appear in between the initialize and finalize calls.
\end{enumerate}

This model suffers from some disadvantages:
\begin{enumerate}
\item There is no error handling during \indexmpishow{MPI_Init}.
\item MPI can not be finalized and restarted;
\item If multiple libraries are active, they can not initialize
  or finalize MPI, but have to base themselves on subcommunicators;
  section~\ref{sec:mpi-comm-dup-lib}.
\item There is no threadsafe way of initializing MPI:
  a~library can't safely do
\begin{lstlisting}
MPI_Initialized(&flag);
if (!flag) MPI_Init(0,0);
\end{lstlisting}
if it is running in a multi-threaded environment.
\end{enumerate}

\begin{mpifournote}{Session model}

In addition to the world,
where all MPI is bracketed by \indexmpishow{MPI_Init}
(or \indexmpishow{MPI_Init_thread}) and \indexmpishow{MPI_Finalize},
there is the \indexterm{session model},
where entities such as libraries can start/end their MPI session
independently.

The two models can be used in the same program, but there are limitations
on how they can mix.

\Level 1 {Short description of the session model}

In the \indextermdef{session model},
each session starts and finalizes MPI independently,
giving each a separate \indexmpishow{MPI_COMM_WORLD}.
The world model then becomes a separate way of starting MPI.
You can create a communicator using the world model
in addition to starting multiple sessions,
each on
their own set of processes, possibly identical or overlapping.
You can also create sessions without have an \indexmpishow{MPI_COMM_WORLD}
created by the world model.

You can not mix in a single call objects
from different sessions,
from a session and from the world model,
or from a session and from \indexmpishow{MPI_Comm_get_parent}
or \indexmpishow{MPI_Comm_join}.

\Level 1 {Session creation}

An MPI \indextermdef{session} is initialized and finalized
with \indexmpiref{MPI_Session_init} and \indexmpidef{MPI_Session_finalize},
somewhat similar to \indexmpishow{MPI_Init} and \indexmpishow{MPI_Finalize}.

\cverbatimsnippet{sessioninit}

This call is thread-safe, in view of the above reasoning.

\Level 2 {Session info}

The \indexmpishow{MPI_Info} object that is passed to \indexmpishow{MPI_Session_init}
can be null, or
it can be used to request a threading level:
%
\cverbatimsnippet{sessionreq}
%
Other info keys can be implementation-dependent,
but the key \indexmpishow{thread_support} is pre-defined.

Info keys can be retrieved again with \indexmpidef{MPI_Session_get_info}:
%
\cverbatimsnippet{sessioninfohave}

\Level 2 {Session error handler}

The error handler argument accepts a pre-defined error handler
(section~\ref{sec:mpi-error})
or one created by \indexmpidef{MPI_Session_create_errhandler}.

\Level 1 {Process sets and communicators}

A session has a number of \indextermbus{process}{set}s.
Process sets are indicated with a \acf{URI},
where the \acp{URI} \indexmpishow{mpi://WORLD} and \indexmpishow{mpi://SELF}
are always defined.

You query the `psets' with \indexmpidef{MPI_Session_get_num_psets}
and \indexmpidef{MPI_Session_get_nth_pset}:
%
\csnippetwithoutput{sessionpsetq}{examples/mpi/c}{session}

The following partial code creates a communicator equivalent to \indexmpishow{MPI_COMM_WORLD}
in the session model:
%
\cverbatimsnippet{sessioncommworld}

However, comparing communicators (with \indexmpishow{MPI_Comm_compare})
from the session and world model,
or from different sessions,
is undefined behavior.

\begin{comment}
  The resulting communicator compares as \indexmpishow{MPI_CONGRUENT}
  under \indexmpishow{MPI_Comm_compare}:
  %
  \csnippetwithoutput{sessionmix}{examples/mpi/c}{sessionmix}
\end{comment}

Get the info object (section~\ref{sec:mpi-info}) from a process set:
\indexmpishow{MPI_Session_get_pset_info}.
This info object always has the key \indexmpishow{mpi_size}.

\Level 1 {Example}

As an example of the use of sessions, we declare a library class,
where each library object starts and ends its own session:
%
\cverbatimsnippet{sessionlibdef}

Now we create a main program, using the world model,
which activates two libraries,
passing data to them by parameter:
%
\cverbatimsnippet{sessionlibuse}
%
Note that no mpi calls will go between main program and either
of the libraries,
or between the two libraries,
but this seems to make sense in this scenario.

\begin{comment}
7.2.4 When using the Sessions Model (Section 11.3) for initialization of MPI
re- sources, MPI_COMM_WORLD and MPI_COMM_SELF are not valid for use as
a communica- tor.

11.3 MPI objects derived from the Sessions Model shall not be
intermixed in a single MPI procedure call with MPI objects derived
from the World Model.

Are those two at odds? The first one seems to imply that in the second
statement objects from the world model can't even exist.
\end{comment}

\end{mpifournote}

\Level 0 {Functionality available outside init/finalize}

\begin{raggedlist}
\indexmpishow{MPI_Initialized}
\indexmpishow{MPI_Finalized}
\indexmpishow{MPI_Get_version}
\indexmpishow{MPI_Get_library_version}
\indexmpishow{MPI_Info_create}
\indexmpishow{MPI_Info_create_env}
\indexmpishow{MPI_Info_set}
\indexmpishow{MPI_Info_delete}
\indexmpishow{MPI_Info_get}
\indexmpishow{MPI_Info_get_valuelen}
\indexmpishow{MPI_Info_get_nkeys}
\indexmpishow{MPI_Info_get_nthkey}
\indexmpishow{MPI_Info_dup}
\indexmpishow{MPI_Info_free}
\indexmpishow{MPI_Info_f2c}
\indexmpishow{MPI_Info_c2f}
\indexmpishow{MPI_Session_create_errhandler}
\indexmpishow{MPI_Session_call_errhandler}
\indexmpishow{MPI_Errhandler_free}
\indexmpishow{MPI_Errhandler_f2c}
\indexmpishow{MPI_Errhandler_c2f}
\indexmpishow{MPI_Error_string}
\indexmpishow{MPI_Error_class}
\end{raggedlist}
Also all routines starting with \indexmpishow{MPI_Txxx}.
