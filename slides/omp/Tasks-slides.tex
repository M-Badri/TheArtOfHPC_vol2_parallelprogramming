% -*- latex -*-
%%%%%%%%%%%%%%%%%%%%%%%%%%%%%%%%%%%%%%%%%%%%%%%%%%%%%%%%%%%%%%%%
%%%%%%%%%%%%%%%%%%%%%%%%%%%%%%%%%%%%%%%%%%%%%%%%%%%%%%%%%%%%%%%%
%%%%
%%%% This text file is part of the lecture slides for
%%%% `Parallel Programming in MPI and OpenMP'
%%%% by Victor Eijkhout, copyright 2012-2024
%%%%
%%%% Tasks-slides.tex : slides about OpenMP workshare constructs
%%%%
%%%%%%%%%%%%%%%%%%%%%%%%%%%%%%%%%%%%%%%%%%%%%%%%%%%%%%%%%%%%%%%%
%%%%%%%%%%%%%%%%%%%%%%%%%%%%%%%%%%%%%%%%%%%%%%%%%%%%%%%%%%%%%%%%

\begin{numberedframe}{Tasks}
  \begin{itemize}
  \item Loops generate `implicit tasks'\\
    need static number of iterations
  \item Also: `explicit tasks' \\
    can deal with dynamic work:
  \item 
    Example linked lists or trees
  \item Tasks are very flexible:\\
    you create work, it goes on a queue, gets executed later
  \end{itemize}
\begin{lstlisting}
p = head_of_list();
while (!end_of_list(p)) {
#pragma omp task
  process( p );
  p = next_element(p);
}
\end{lstlisting}
\end{numberedframe}

\begin{numberedframe}{Threads, tasks, queues}
  \begin{itemize}
  \item There is one queue (per team), not visible to the programmer.
  \item One thread starts generating tasks.
  \item Tasks can recursively generate tasks.
  \item You never know who executes what.
  \end{itemize}
  \includegraphics[scale=.1]{taskthread}
\end{numberedframe}

\begin{numberedframe}{Generation idiom}
\begin{lstlisting}
#pragma omp parallel
#pragma omp single
{
  ...
  #pragma omp task
  { ... }
  #pragma omp taskwait
}
\end{lstlisting}
\end{numberedframe}

\begin{numberedframe}{Recursive generation}
\begin{lstlisting}
void f() {
  do_something();
  #pragma omp task
  do_task();
  do_some_more();
}
...
#pragma omp parallel
#pragma omp single
#pragma omp task
  f();
\end{lstlisting}
  \begin{itemize}
  \item Any thread can create tasks
  \item including threads that are executing tasks
  \end{itemize}
\end{numberedframe}

\begin{exerciseframe}[taskfactor]
  \input ex:taskfactor
\end{exerciseframe}

\begin{numberedframe}{Task data space}
  \begin{itemize}
  \item Environment captured as \lstinline{firstprivate}
    unless explicitly otherwise
  \item Makes sense: execution is deferred
\begin{lstlisting}
#pragma omp parallel
#pragma omp single
for ( int i /* ... */ )
  #pragma omp task
  f(i);
\end{lstlisting}
  \item Shared data has to be captured explicitly
  \end{itemize}
\end{numberedframe}

\begin{numberedframe}{Task data: tricky exception}
\begin{lstlisting}
#pragma omp parallel shared(i)
#pragma omp single
for ( i /* ... */ ) // note: no `int'
  #pragma omp task firstprivate(i)
  f(i);
\end{lstlisting}
\end{numberedframe}

\begin{numberedframe}{Task synchronization}
  Mechanisms for task synchronization:
  \begin{itemize}
  \item \indextermtt{taskwait}: wait for all previous tasks (not nested)
  \item \indextermtt{taskgroup}: wait for all tasks, including nested
  \item \indextermtt{depend}: synchronize on data items.
  \end{itemize}
\end{numberedframe}

\begin{numberedframe}{Task wait}
Insider parallel region:
\begin{lstlisting}
#pragma omp task
...
#pragma omp task
...
#pragma omp taskwait
\end{lstlisting}
  \begin{itemize}
  \item Wait on tasks spawned directly
  \item Not `descendant' tasks
  \end{itemize}
\end{numberedframe}

\begin{numberedframe}{Task group}
\begin{lstlisting}
#pragma omp taskgroup
{
  #pragma omp task
  foo(0);
  #pragma omp task
  for ( int i=1; i<N; ++i )
    #pragma omp task
    foo(i)
}
\end{lstlisting}
  \begin{itemize}
  \item Wait for tasks on this level
  \item \ldots~and tasks created by tasks on this level
  \end{itemize}
\end{numberedframe}

\begin{numberedframe}{Example: tree traversal}
\begin{lstlisting}
int process( node n ) {
  if (n.is_leaf)
    return n.value;
  for ( c : n.children) {
    #pragma omp task
    process(c);
#pragma omp taskwait
  return sum_of_children();
}
\end{lstlisting}
\end{numberedframe}

\begin{numberedframe}{Exercise: Fibonacci}
  Fix the errors in this program.
\begin{lstlisting}
long fib(int n) {
  if (n<2) return n;
  else { long f1,f2;
    #pragma omp task
    f1 = fib(n-1);
    #pragma omp task
    f2 = fib(n-2);
    return f1+f2;
  }
}

#pragma omp parallel
#pragma omp single
  printf("Fib(50)=%ld",fib(50));
\end{lstlisting}
\end{numberedframe}

\begin{numberedframe}{Task dependencies}
\begin{multicols}{2}
\begin{lstlisting}
#pragma omp task
  x = f()
#pragma omp task
  y = g(x)
\end{lstlisting}
\columnbreak
\footnotesize
\begin{lstlisting}
#pragma omp task depend(out:x)
  x = f()
#pragma omp task depend(in:x)
  y = g(x)
\end{lstlisting}
\end{multicols}

 Sequence tasks by indicating
  \begin{itemize}
  \item The output of one task
  \item the needed input of another
  \item (only between sibling tasks)
  \end{itemize}
\end{numberedframe}

\begin{numberedframe}{Exercise: dependencies}
\begin{lstlisting}
for i in [1:N]:
    x[0,i] = some_function_of(i)
    x[i,0] = some_function_of(i)

for i in [1:N]:
    for j in [1:N]:
        x[i,j] = x[i-1,j]+x[i,j-1]
\end{lstlisting}
  \begin{itemize}
  \item Use tasks
  \item Is there a different way to do this?
  \end{itemize}
\end{numberedframe}

\begin{numberedframe}{Fibonacci memo'ization}
\begin{lstlisting}
long fibvalues[100];
void fibonacci(n) {
  if (n>=2) {
    #pragma omp task \
       depend( in:fibvalues[n-2],in:fibvalues[n-1] ) \
       depend( out:fibvalues[n] )
    fibvalues[n] = fibvalues[n-2]+fibvalues[n-1];
};

#pragma omp parallel
#pragma omp single
  for (i<50)
    fibonacci(i);
\end{lstlisting}
\end{numberedframe}


\begin{numberedframe}{Taskloop}
\begin{lstlisting}
#pragma omp taskloop grainsize(5)
for ( int i=0; i<N; ++i )
  f(i)
\end{lstlisting}
  \begin{itemize}
  \item Turn iterations into tasks
  \item Implicit taskgroup, unless \lstinline{nogroup} specified
  \end{itemize}
\end{numberedframe}

\endinput

\begin{numberedframe}{}
\begin{lstlisting}
\end{lstlisting}
  \begin{itemize}
  \item 
  \end{itemize}
\end{numberedframe}

