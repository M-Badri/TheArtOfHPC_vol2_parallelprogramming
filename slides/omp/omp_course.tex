% -*- latex -*-
%%%%%%%%%%%%%%%%%%%%%%%%%%%%%%%%%%%%%%%%%%%%%%%%%%%%%%%%%%%%%%%%
%%%%%%%%%%%%%%%%%%%%%%%%%%%%%%%%%%%%%%%%%%%%%%%%%%%%%%%%%%%%%%%%
%%%%
%%%% This text file is part of the source of 
%%%% `Parallel Programming in MPI and OpenMP'
%%%% by Victor Eijkhout, copyright 2012-2024
%%%%
%%%% omp_course.tex : master file for an OpenMP course
%%%%
%%%%%%%%%%%%%%%%%%%%%%%%%%%%%%%%%%%%%%%%%%%%%%%%%%%%%%%%%%%%%%%%
%%%%%%%%%%%%%%%%%%%%%%%%%%%%%%%%%%%%%%%%%%%%%%%%%%%%%%%%%%%%%%%%

\documentclass[10pt]{beamer}

\input courseformat
\usepackage{pdfpages}

\newcounter{chapter} % how is this handled in ISP?
\input blockmacs

\def\qrcode{qrvol2}
\begin{document}
\input lang

\author[Eijkhout]{Victor Eijkhout}
\date[2024]{2024 COE 379L / CSE 392}
%% \normalsize last formatted \today}
\title[OMP]{OpenMP Course}
\maketitle

\begin{frame}{Justification}
  OpenMP is a flexible tool for incrementally parallelizing a shared
  memory-based code.
  This course introduces the main concepts
  through lecturing and exercises.
\end{frame}

\Level 0 {The Fork-Join model}
\input Forkjoin-slides

\Level 0 {Loop parallelism}
\input ParLoop-slides

\Level 0 {Workshare constructs}
\input Workshare-slides

\Level 0 {Thread data}
\input OmpData-slides

\Level 0 {Synchronization}
\input Sync-slides

\Level 0 {Tasks}
\input Tasks-slides

\Level 0 {Memory model}
\input Memory-slides

\Level 0 {Remaining topics}
\input OmpOther-slides

\end{document}

\newenvironment
    {theindex}
    {\begin{itemize}\setlength\itemsep{0pt}\baselineskip=8pt}
    {\end{itemize}}
\let\indexspace\par
\def\subitem{\par\indent}
%
\begin{frame}{Index}
\small
\begin{multicols}{2}
\printindex  
\end{multicols}
\end{frame}

\begin{frame}{Bibliography}
  \bibliographystyle{plain}
  \bibliography{vle}
\end{frame}

\end{document}
