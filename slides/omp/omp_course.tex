% -*- latex -*-
%%%%%%%%%%%%%%%%%%%%%%%%%%%%%%%%%%%%%%%%%%%%%%%%%%%%%%%%%%%%%%%%
%%%%%%%%%%%%%%%%%%%%%%%%%%%%%%%%%%%%%%%%%%%%%%%%%%%%%%%%%%%%%%%%
%%%%
%%%% This text file is part of the source of 
%%%% `Parallel Programming in MPI and OpenMP'
%%%% by Victor Eijkhout, copyright 2012-2023
%%%%
%%%% omp_course.tex : master file for an OpenMP course
%%%%
%%%%%%%%%%%%%%%%%%%%%%%%%%%%%%%%%%%%%%%%%%%%%%%%%%%%%%%%%%%%%%%%
%%%%%%%%%%%%%%%%%%%%%%%%%%%%%%%%%%%%%%%%%%%%%%%%%%%%%%%%%%%%%%%%

\documentclass[11pt,headernav]{beamer}

\beamertemplatenavigationsymbolsempty
\usetheme{Madrid}%{Montpellier}
\usecolortheme{seahorse}
\setcounter{tocdepth}{1}
%% \AtBeginSection[]
%% {
%%   \begin{frame}
%%     \frametitle{Table of Contents}
%%     \tableofcontents[currentsection]
%%   \end{frame}
%% }

%Global Background must be put in preamble
%\usebackgroundtemplate{\includegraphics[width=\paperwidth,height=\paperheight]{newton.jpg}}

\setbeamertemplate{footline}{\hskip1em Eijkhout: OMP intro\hfill
  \hbox to 0in {\hss \includegraphics[scale=.1]{tacclogonew}}%
  \hbox to 0in {\hss \arabic{page}\hskip 1in}}

\usepackage{morewrites}

\usepackage{comment}
\input slides.inex

\input commonmacs
\input acromacs
\input slidemacs
\input coursemacs
\input snippetmacs

\begin{document}
\parskip=10pt plus 5pt minus 3pt

\title{Tutorial on OpenMP programming}
\author{Victor Eijkhout}
\date{SSiASC 2016}

\begin{frame}
  \titlepage
\end{frame}

\begin{frame}{Justification}
  OpenMP is a flexible tool for incrementally parallelizing a shared
  memory-based code.
  This course introduces the main concepts
  through lecturing and exercises.
\end{frame}

\Level 0 {The Fork-Join model}
\input Forkjoin-slides

\Level 0 {Loop parallelism}
\input ParLoop-slides

\Level 0 {Workshare constructs}
\input Workshare-slides

\Level 0 {Thread data}
\input OmpData-slides

\Level 0 {Synchronization}
\input Sync-slides

\Level 0 {Tasks}
\input Tasks-slides

\Level 0 {Remaining topics}
\input OmpOther-slides

\end{document}

\newenvironment
    {theindex}
    {\begin{itemize}\setlength\itemsep{0pt}\baselineskip=8pt}
    {\end{itemize}}
\let\indexspace\par
\def\subitem{\par\indent}
%
\begin{frame}{Index}
\small
\begin{multicols}{2}
\printindex  
\end{multicols}
\end{frame}

\begin{frame}{Bibliography}
  \bibliographystyle{plain}
  \bibliography{vle}
\end{frame}

\end{document}
