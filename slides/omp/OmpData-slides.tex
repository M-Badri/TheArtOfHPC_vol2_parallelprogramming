% -*- latex -*-
%%%%%%%%%%%%%%%%%%%%%%%%%%%%%%%%%%%%%%%%%%%%%%%%%%%%%%%%%%%%%%%%
%%%%%%%%%%%%%%%%%%%%%%%%%%%%%%%%%%%%%%%%%%%%%%%%%%%%%%%%%%%%%%%%
%%%%
%%%% This text file is part of the lecture slides for
%%%% `Parallel Programming in MPI and OpenMP'
%%%% by Victor Eijkhout, copyright 2012-2024
%%%%
%%%% OmpData-slides.tex : slides about OpenMP workshare constructs
%%%%
%%%%%%%%%%%%%%%%%%%%%%%%%%%%%%%%%%%%%%%%%%%%%%%%%%%%%%%%%%%%%%%%
%%%%%%%%%%%%%%%%%%%%%%%%%%%%%%%%%%%%%%%%%%%%%%%%%%%%%%%%%%%%%%%%

\begin{numberedframe}{Shared and private data}
  You have already seen some of the basics:
  \begin{itemize}
  \item Data declared outside a parallel region is shared.
  \item Data declared in the parallel region is private.\\
    (Fortran does not have this block scope mechanism)
\begin{lstlisting}
int i;
#pragma omp parallel 
{ double i; .... }
\end{lstlisting}
  \item You can change all this with clauses:
\begin{lstlisting}
int i;
#pragma omp parallel private(i)
\end{lstlisting}
  \end{itemize}
\end{numberedframe}

\begin{numberedframe}{Variables in loops}
\begin{lstlisting}
int i; double t;
#pragma omp parallel for 
  for (i=0; i<N; i++) {
    t = sin(i*pi*h);
    x[i] = t*t;
  }
\end{lstlisting}
  \begin{itemize}
  \item The loop variable is automatically private.
  \item The temporary \n{t} is shared, but conceptually private to
    each iteration: needs to be declared private.\\
    (What happens if you don't?)
  \end{itemize}
\end{numberedframe}

\begin{numberedframe}{Copying to/from private data}
  \begin{itemize}
  \item Private data is uninitialized
\begin{lstlisting}
int i = 3;
#pragma omp parallel private(i)
  printf("%d\n",i); // undefined!
\end{lstlisting}
  \item To import a value:
\begin{lstlisting}
int i = 3;
#pragma omp parallel firstprivate(i)
  printf("%d\n",i); // undefined!
\end{lstlisting}
  \item \indextermtt{lastprivate} to preserve value of last iteration.
  \end{itemize}
\end{numberedframe}

\begin{numberedframe}{Default behaviour}
  \begin{itemize}
  \item \indextermtt{default(shared)} or \indextermtt{default(private)}
  \item useful for debugging: \indextermtt{default(none)}\\ because you have
    to specify everything as shared/private
  \end{itemize}
\end{numberedframe}

\begin{numberedframe}{Persistent thread data}
  \begin{itemize}
  \item Private data disappears after the parallel region.\\
    What if you want data to persist?
  \item Directive \indextermtt{threadprivate}
\begin{lstlisting}
double seed;
#pragma omp threadprivate(seed)
\end{lstlisting}
\item Standard application: random number generation.
\item Tricky: has to be global or static.
  \end{itemize}
\end{numberedframe}

\begin{numberedframe}{Arrays}
  \begin{itemize}
  \item Statically allocated arrays can be made private.
  \item Dynamically allocated ones can not: the pointer becomes private.
  \end{itemize}
\end{numberedframe}

\endinput

\begin{numberedframe}{}
  \begin{itemize}
  \item 
  \end{itemize}
\end{numberedframe}

