% -*- latex -*-
%%%%%%%%%%%%%%%%%%%%%%%%%%%%%%%%%%%%%%%%%%%%%%%%%%%%%%%%%%%%%%%%
%%%%%%%%%%%%%%%%%%%%%%%%%%%%%%%%%%%%%%%%%%%%%%%%%%%%%%%%%%%%%%%%
%%%%
%%%% This text file is part of the source of 
%%%% `Parallel Computing'
%%%% by Victor Eijkhout, copyright 2012-2023
%%%%
%%%% mpibasics_course.tex : basic part of the MPI course
%%%%
%%%%%%%%%%%%%%%%%%%%%%%%%%%%%%%%%%%%%%%%%%%%%%%%%%%%%%%%%%%%%%%%
%%%%%%%%%%%%%%%%%%%%%%%%%%%%%%%%%%%%%%%%%%%%%%%%%%%%%%%%%%%%%%%%

\documentclass[11pt,headernav]{beamer}

\beamertemplatenavigationsymbolsempty
\usetheme{Madrid}%{Montpellier}
\usecolortheme{seahorse}
\setcounter{tocdepth}{1}

\setbeamertemplate{footline}{\hskip1em Eijkhout: MPI course\hfill
  \hbox to 0in {\hss \includegraphics[scale=.1]{tacclogonew}}%
  \hbox to 0in {\hss \arabic{page}\hskip 1in}}

\usepackage{morewrites}

\usepackage{comment}
\input slides.inex
\input commonmacs
\input acromacs
\input slidemacs
\input coursemacs
\input snippetmacs

\input doxymacs

%%%%
%%%% Where is this course?
%%%%

\def\Location{}% redefine in the inex file
\def\courseyear{2023}
%\def\Location{TACC HPC Training \courseyear}
\def\Location{SDS PCSE \courseyear}

\includecomment{full}
\excludecomment{condensed}
\excludecomment{online}

\includecomment{utonly}

\includecomment{onesided}
\includecomment{advanced}
\includecomment{foundations}
\excludecomment{solutions}

\def\TitleExtra{\\ Basic material}

%%%%%%%%%%%%%%%%
%%%%%%%%%%%%%%%% Document
%%%%%%%%%%%%%%%%
\input lang

\begin{document}
\parskip=10pt plus 5pt minus 3pt

\title{Tutorial on MPI programming\TitleExtra}
\author{Victor Eijkhout {\tt eijkhout@tacc.utexas.edu}}
\date{\Location}

\begin{frame}
  \titlepage
\end{frame}

\begin{frame}[containsverbatim]{Materials}
    Textbooks and repositories:\\
    \url{https://theartofhpc.com}
\end{frame}

\begin{frame}{Justification}
  The MPI library is the main tool
  for parallel programming on a large scale.
  This course introduces the main concepts
  through lecturing and exercises.
\end{frame}

\begin{frame}{Table of Contents}
  \def\contentsline####1####2{\item{} ####2}
  \IfFileExists{\jobname.toc}
               {\begin{itemize}
                   \tableofcontents
               \end{itemize}
               }{}
\end{frame}

\coursepart{Basics}

\renewcommand\standardversion{3}

\Level 0 {The SPMD model}
\input SPMD-slides

\Level 0 {Collectives}
\input Collective-slides

\Level 0 {Point-to-point communication}
\input PTP-slides

\iffalse
\begin{exerciseframe}[serialsend]
  \input ex:serialsend
\end{exerciseframe}
\fi

\begin{frame}[containsverbatim]\frametitle{Where to go from here\ldots}
  \begin{itemize}
  \item Derived data types: send strided/irregular/inhomogeneous data
  \item Sub-communicators: work with subsets of \indexmpishow{MPI_COMM_WORLD}
  \item I/O: efficient file operations
  \item One-sided communication: `just' put/get the data somewhere
  \item Process management
  \item Non-blocking collectives
  \item Graph topology and neighborhood collectives
  \item Shared memory
  \end{itemize}
\end{frame}

\renewcommand\standardversion{}

\end{document}

