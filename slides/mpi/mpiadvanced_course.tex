% -*- latex -*-
%%%%%%%%%%%%%%%%%%%%%%%%%%%%%%%%%%%%%%%%%%%%%%%%%%%%%%%%%%%%%%%%
%%%%%%%%%%%%%%%%%%%%%%%%%%%%%%%%%%%%%%%%%%%%%%%%%%%%%%%%%%%%%%%%
%%%%
%%%% This text file is part of the source of 
%%%% `Parallel Computing'
%%%% by Victor Eijkhout, copyright 2012-2023
%%%%
%%%% mpi_course.tex : master file for an advanced MPI course
%%%%
%%%%%%%%%%%%%%%%%%%%%%%%%%%%%%%%%%%%%%%%%%%%%%%%%%%%%%%%%%%%%%%%
%%%%%%%%%%%%%%%%%%%%%%%%%%%%%%%%%%%%%%%%%%%%%%%%%%%%%%%%%%%%%%%%

\documentclass[10pt]{beamer}

\input courseformat

\def\qrcode{qrvol2}
\begin{document}

\title[MPI 3\&4]{Advanced Features of MPI-3 and MPI-4}
\author[Eijkhout]{Victor Eijkhout}
\date{PEARC 2023}

\maketitle

\begin{frame}[containsverbatim]{Materials}
  Textbooks and repositories:\\
  \url{https://theartofhpc.com/pcse}

  Contact:\\
  {\texttt eijkhout@tacc.utexas.edu}
\end{frame}

\begin{frame}{Justification}
  Version 3 of the MPI standard has added a number
  of features, some geared purely towards functionality,
  others with an eye towards efficiency at exascale.

  Version 4 adds yet more features for exascale,
  and more flexible process management.

  \begin{itemize}
  \item
    MPI-3 as of 2012, 3.1 as of~2015. Fully supported everywhere.\\
  \item 
    MPI-4 as of June 2021;
    Supported in mpich version~4.1, \emph{not} in OpenMPI version~4.
  \item MPI-4.1 to be ratified late 2023.
  \end{itemize}
\end{frame}

\Level 0 {Fortran bindings}
\input F08_slides

\Level 0 {Big data communication}
\input Bigdata-slides

\Level 0 {Advanced collectives}
\input Highercollective-slides

\Level 0 {Shared memory}
\input Sharedmemory-slides

\Level 0 {Atomic operations}
\input Atomic-slides
 
\Level 0 {Partitioned communication}
\input Partitioned-slides

\Level 0 {Sessions model}
\input Sessions-slides

\Level 0 {Process topologies}
\input Graph-slides

\Level 0 {Other MPI-4 material}

\begin{numberedframe}{Better aborts}
  \begin{itemize}
  \item Error handler
     \indexmpidef{MPI_ERRORS_ABORT}:
     aborts on the processes in the communicator for which it is specified.
  \item
    Error code \indexmpidef{MPI_ERR_PROC_ABORTED}:
    process tried to communicate
    with a process that has aborted.
  \end{itemize}

\end{numberedframe}

\begin{numberedframe}{Error as C-string}
 \indexmpidepr{MPI_Info_get} and \indexmpidepr{MPI_Info_get_valuelen}
 are not robust with respect to the  \indexterm{null terminator}.\\
 Replace by:
\begin{lstlisting}
int MPI_Info_get_string
   (MPI_Info info, const char *key, 
    int *buflen, char *value, int *flag)  
\end{lstlisting}
\end{numberedframe}

\Level 0 {Summary}

\begin{numberedframe}{Summary}
  \begin{itemize}
  \item Fortran 2008 bindings (MPI-3)
  \item \indexmpishow{MPI_Count} arguments for large buffers (MPI-4)
  \item Atomic one-sided communication (MPI-3)
  \item Non-blocking collectives (MPI-3) and persistent collectives (MPI-4)
  \item Shared memory (MPI-3)
  \item Graph topologies (MPI-3)
  \item Partitioned sends (MPI-4)
  \item Sessions model (MPI-4)
  \end{itemize}
\end{numberedframe}

\coursepart{Supplemental material}

\begin{exerciseframe}[serialsend]
  \label{exserialsend}
  \input{ex:serialsend}
\end{exerciseframe}

\begin{exerciseframe}[procgrid]
  \input{ex:rowcolcomm}
\end{exerciseframe}

\end{document}

\Level 0 {Appendix: Intercommunicator recap}
\input{sl:comm-inter}
\input{sl:intercomm-picture}
\input{sl:intercomm-concepts}
\input{sl:intercomm-routines}

\begin{comment}
  \begin{numberedframe}{Protocol}
    \label{sl:rendezvous}
    Communication is a `rendez-vous' or `hand-shake' protocol:
    \begin{itemize}
    \item Sender: `I have data for you'
    \item Receiver: `I have a buffer ready, send it over'
    \item Sender: `Ok, here it comes'
    \item Receiver: `Got it.'
    \end{itemize}
    Small messages bypass this: `eager' send.\\
    Definition of `small message' controlled by environment variables.
  \end{numberedframe}
\end{comment}

\end{document}

