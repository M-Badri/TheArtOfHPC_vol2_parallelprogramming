% -*- latex -*-
%%%%%%%%%%%%%%%%%%%%%%%%%%%%%%%%%%%%%%%%%%%%%%%%%%%%%%%%%%%%%%%%
%%%%%%%%%%%%%%%%%%%%%%%%%%%%%%%%%%%%%%%%%%%%%%%%%%%%%%%%%%%%%%%%
%%%%
%%%% This text file is part of the lecture slides for
%%%% `Parallel Computing'
%%%% by Victor Eijkhout, copyright 2018-2021
%%%%
%%%% F08-slides.tex : about Fortran 2008 bindings
%%%%
%%%%%%%%%%%%%%%%%%%%%%%%%%%%%%%%%%%%%%%%%%%%%%%%%%%%%%%%%%%%%%%%
%%%%%%%%%%%%%%%%%%%%%%%%%%%%%%%%%%%%%%%%%%%%%%%%%%%%%%%%%%%%%%%%

\lstset{language=Fortran}
\begin{numberedframe}{Overview}
  The Fortran interface to MPI had some defects.
  With Fortran2008 these have been largely repaired.
  \begin{itemize}
  \item The trailing error parameter is now optional;
  \item MPI data types are now actual \lstinline{Type} objects,
    rather than \lstinline{Integer}
  \item Strict type checking on arguments.
  \end{itemize}
\end{numberedframe}

\begin{numberedframe}{MPI headers}
\label{sl:mpi-header}
New module:
\begin{verbatim}
use mpi_f08   ! for Fortran2008
use mpi       ! for Fortran90
\end{verbatim}
True Fortran bindings as of the 2008 standard.
\begin{tacc}
  Provided in
  \begin{itemize}
  \item
    Intel compiler version 18 or newer,
  \item 
    gcc~9 and later (not with Intel MPI, use mvapich).
  \end{itemize}
\end{tacc}
\end{numberedframe}

\begin{numberedframe}{Optional error parameter}
\lstset{language=Fortran}
\begin{multicols}{2}
  Old Fortran90 style:
\begin{lstlisting}
call MPI_Init(ierr)
! your code
call MPI_Finalize(ierr)
\end{lstlisting}
\columnbreak
New Fortran2008 style:
\begin{lstlisting}
call MPI_Init()
! your code
call MPI_Finalize()
\end{lstlisting}
\end{multicols}
\end{numberedframe}

\begin{numberedframe}{Communicators}
Communicators are now derived types:
\begin{lstlisting}
!! Fortran 2008 interface
use mpi_f08
Type(MPI_Comm) :: comm = MPI_COMM_WORLD
\end{lstlisting}
\begin{lstlisting}
!! Fortran legacy interface
use mpi
!! deprecated: #include <mpif.h>
Integer :: comm = MPI_COMM_WORLD
\end{lstlisting}
\end{numberedframe}

\begin{numberedframe}{Requests}
  Requests are also derived types\\
  note that \n{...NULL} entities are now objects, not integers

  \fverbatimsnippet{waitany0f}

  (Q for the alert student: do you see anything halfway
  remarkable about that index?)
\end{numberedframe}

\begin{numberedframe}{More}
  More handles that are now derived types:
\begin{lstlisting}
Type(MPI_Datatype) :: newtype ! F2008
Integer            :: newtype ! F90
\end{lstlisting}

Also:
  \lstinline{MPI_Comm}, \lstinline{MPI_Datatype},
  \lstinline{MPI_Errhandler}, \lstinline{MPI_Group},
  \lstinline{MPI_Info}, \lstinline{MPI_File}, \lstinline{MPI_Message},
            \lstinline{MPI_Op},
  \lstinline{MPI_Request}, \lstinline{MPI_Status}, \lstinline{MPI_Win}
\end{numberedframe}

\begin{numberedframe}{Status}
  Fortran2008: status is a \lstinline{Type} with fields:
  \fverbatimsnippet{anysource-f08}

  Fortran90: status is an array with named indexing
  \fverbatimsnippet{anysource-f}
\end{numberedframe}

\begin{numberedframe}{Type checking}
  Type checking catches potential problems:
  \fverbatimsnippet{ftypecheckarg}
\begin{verbatim}
typecheck.F90(20): error #6285: 
  There is no matching specific subroutine
  for this generic subroutine call.   [MPI_SEND]
  call MPI_Send(source,MPI_INTEGER,n,
-------^
\end{verbatim}
\end{numberedframe}

\begin{numberedframe}{Type checking'}
  Type checking does not catch all problems:
  \fverbatimsnippet{ftypecheckbuf}
  Buffer/type mismatch is not caught.
\end{numberedframe}

\lstset{language=C}
