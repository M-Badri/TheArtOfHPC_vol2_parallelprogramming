% -*- latex -*-
%%%%%%%%%%%%%%%%%%%%%%%%%%%%%%%%%%%%%%%%%%%%%%%%%%%%%%%%%%%%%%%%
%%%%%%%%%%%%%%%%%%%%%%%%%%%%%%%%%%%%%%%%%%%%%%%%%%%%%%%%%%%%%%%%
%%%%
%%%% This text file is part of the source of 
%%%% `Parallel Computing'
%%%% by Victor Eijkhout, copyright 2012-2024
%%%%
%%%% mpi_course.tex : master file for an MPI course
%%%%
%%%%%%%%%%%%%%%%%%%%%%%%%%%%%%%%%%%%%%%%%%%%%%%%%%%%%%%%%%%%%%%%
%%%%%%%%%%%%%%%%%%%%%%%%%%%%%%%%%%%%%%%%%%%%%%%%%%%%%%%%%%%%%%%%

\documentclass[10pt]{beamer}

\input courseformat
  
\def\qrcode{qrvol2}
\begin{document}
\input lang

\author[Eijkhout]{Victor Eijkhout}
\date[2024]{2024 COE 379L / CSE 392}
%% \normalsize last formatted \today}
\title[MPI]{MPI Course}
\maketitle

%% \event{2024 COE 379L / CSE 392}
%% \author{Victor Eijkhout}
%% \date{Spring 2024\\ \tiny\texttt{last formatted: \today}}

%% \newcommand\titleslide[1]{
%%   \title{#1}
%%   \date{\event}
%%   \frame{\titlepage}
%% }
%% \titleslide{Parallel programming in MPI}

\begin{frame}[containsverbatim]{Materials}
    Textbooks and repositories:\\
    \url{https://theartofhpc.com}
\end{frame}

\begin{frame}{Justification}
  The MPI library is the main tool
  for parallel programming on a large scale.
  This course introduces the main concepts
  through lecturing and exercises.
\end{frame}

\begin{frame}{Table of Contents}
  \begin{enumerate}
  \item The SPMD Model~\hyperlink{sec:spmd}{\textsl{link}}
  \item Collectives~\hyperlink{sec:collectives}{\textsl{link}}
  \item Point-to-point~\hyperlink{sec:ptp}{\textsl{link}}
  \item Derived datatypes~\hyperlink{sec:derived}{\textsl{link}}
  \item Communicators~\hyperlink{sec:comm}{\textsl{link}}
  \item MPI I/O~\hyperlink{sec:io}{\textsl{link}}
  \item One-sided communication~\hyperlink{sec:1side}{\textsl{link}}
  \item Big data~\hyperlink{sec:bigdata}{\textsl{link}}
  \item Advanced collectives~\hyperlink{sec:coll2}{\textsl{link}}
  \item Shared memory~\hyperlink{sec:shared}{\textsl{link}}
  \item Process management~\hyperlink{sec:process}{\textsl{link}}
  \item Process topologies~\hyperlink{sec:topo}{\textsl{link}}
  \item Trace and performance~\hyperlink{sec:trace}{\textsl{link}}
  \end{enumerate}
\end{frame}

\coursepart{Basics}

\renewcommand\standardversion{3}

\coursesection{sec:spmd}{The SPMD model}
\input SPMD-slides

\coursesection{sec:collectives}{Collectives}
\input Collective-slides

\coursesection{sec:ptp}{Point-to-point communication}
\input PTP-slides

\iffalse
\begin{exerciseframe}[serialsend]
  \input ex:serialsend
\end{exerciseframe}
\fi

\begin{frame}[containsverbatim]\frametitle{Where to go from here\ldots}
  \begin{itemize}
  \item Derived data types: send strided/irregular/inhomogeneous data
  \item Sub-communicators: work with subsets of \indexmpishow{MPI_COMM_WORLD}
  \item I/O: efficient file operations
  \item One-sided communication: `just' put/get the data somewhere
  \item Process management
  \item Non-blocking collectives
  \item Graph topology and neighborhood collectives
  \item Shared memory
  \end{itemize}
\end{frame}

\renewcommand\standardversion{}

\coursepart{Intermediate topics}

\begin{frame}{Justification}
  MPI basic concepts suffice for many applications.  The Intermediate
  Topics section deals with more complicated data, process groups,
  file I/O, and the basics of one-sided communication.
\end{frame}

\coursesection{sec:derived}{Derived Datatypes}
\input Data-slides

\coursesection{sec:comm}{Communicator manipulations}
\input Subcomm-slides

\coursesection{sec:io}{MPI File I/O}
\input MPIO-slides

\coursesection{sec:1side}{One-sided communication}
\input Onesided-slides 
\input Atomic-slides

\coursesection{sec:bigdata}{Big data communication}
\input Bigdata-slides

\coursepart{Advanced (MPI-3/4) topics}

\begin{frame}{Justification}
  Recent additions to the MPI standard allow your 
  code to deal with unusual scenarios or very large scale runs.
\end{frame}

\coursesection{sec:coll2}{Advanced collectives}
\input Highercollective-slides

\coursesection{sec:shared}{Shared memory}
\input Sharedmemory-slides

\coursesection{sec:process}{Process management}
\input Spawn-slides

\coursesection{sec:topo}{Process topologies}
\input Graph-slides

\coursepart{Other}

\coursesection{sec:trace}{Tracing, performance, and such}
\input Performance-slides

\end{document}

