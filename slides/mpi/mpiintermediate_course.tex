% -*- latex -*-
%%%%%%%%%%%%%%%%%%%%%%%%%%%%%%%%%%%%%%%%%%%%%%%%%%%%%%%%%%%%%%%%
%%%%%%%%%%%%%%%%%%%%%%%%%%%%%%%%%%%%%%%%%%%%%%%%%%%%%%%%%%%%%%%%
%%%%
%%%% This text file is part of the source of 
%%%% `Parallel Computing'
%%%% by Victor Eijkhout, copyright 2012-2023
%%%%
%%%% mpiintermediate_course.tex : intermediate MPI topics
%%%%
%%%%%%%%%%%%%%%%%%%%%%%%%%%%%%%%%%%%%%%%%%%%%%%%%%%%%%%%%%%%%%%%
%%%%%%%%%%%%%%%%%%%%%%%%%%%%%%%%%%%%%%%%%%%%%%%%%%%%%%%%%%%%%%%%

\documentclass[11pt,headernav]{beamer}

\beamertemplatenavigationsymbolsempty
\usetheme{Madrid}%{Montpellier}
\usecolortheme{seahorse}
\setcounter{tocdepth}{1}

\setbeamertemplate{footline}{\hskip1em Eijkhout: MPI course\hfill
  \hbox to 0in {\hss \includegraphics[scale=.1]{tacclogonew}}%
  \hbox to 0in {\hss \arabic{page}\hskip 1in}}

%% \usepackage{morewrites}

\usepackage{comment}
\input slides.inex
\input commonmacs
\input acromacs
\input slidemacs
\input coursemacs
\input snippetmacs

\input doxymacs

%%%%
%%%% Where is this course?
%%%%

\def\Location{}% redefine in the inex file
\def\courseyear{2023}
%\def\Location{TACC HPC Training \courseyear}
\def\Location{SDS PCSE \courseyear}

\includecomment{full}
\excludecomment{condensed}
\excludecomment{online}

\includecomment{utonly}

\includecomment{onesided}
\includecomment{advanced}
\excludecomment{solutions}

\def\TitleExtra{}

%%%%%%%%%%%%%%%%
%%%%%%%%%%%%%%%% Document
%%%%%%%%%%%%%%%%
\input lang

\begin{document}
\parskip=10pt plus 5pt minus 3pt

\title{MPI Intermediate Topics\TitleExtra}
\author{Victor Eijkhout {\tt eijkhout@tacc.utexas.edu}}
\date{\Location}

\begin{frame}
  \titlepage
\end{frame}

\begin{frame}[containsverbatim]{Materials}
    Textbooks and repositories:\\
    \url{https://theartofhpc.com}
\end{frame}

%% \begin{frame}{Justification}
%%   The MPI library is the main tool
%%   for parallel programming on a large scale.
%%   This course introduces the main concepts
%%   through lecturing and exercises.
%% \end{frame}

\begin{frame}{Table of Contents}
  \def\contentsline####1####2{\item ####2}
  \IfFileExists{\jobname.toc}
               {\begin{itemize}
                   \tableofcontents
               \end{itemize}
               }{}
\end{frame}

\renewcommand\standardversion{}

\begin{frame}{Justification}
  MPI basic concepts suffice for many applications.  The Intermediate
  Topics section deals with more complicated data, process groups,
  file I/O, and the basics of one-sided communication.
\end{frame}

\Level 0 {Derived Datatypes}
\input Data-slides

\Level 0 {Communicator manipulations}
\input Subcomm-slides

\Level 0 {MPI File I/O}
\input MPIO-slides

\Level 0 {Big data communication}
\input Bigdata-slides

\coursepart{Other}

\Level 0 {Tracing, performance, and such}
\input Performance-slides

\end{document}

