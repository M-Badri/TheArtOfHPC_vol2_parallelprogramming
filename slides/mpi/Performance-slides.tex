% -*- latex -*-
%%%%%%%%%%%%%%%%%%%%%%%%%%%%%%%%%%%%%%%%%%%%%%%%%%%%%%%%%%%%%%%%
%%%%%%%%%%%%%%%%%%%%%%%%%%%%%%%%%%%%%%%%%%%%%%%%%%%%%%%%%%%%%%%%
%%%%
%%%% This text file is part of the lecture slides for
%%%% `Parallel Computing'
%%%% by Victor Eijkhout, copyright 2012-2023
%%%%
%%%% Performance-slides.tex: MPI Performance issues
%%%%
%%%%%%%%%%%%%%%%%%%%%%%%%%%%%%%%%%%%%%%%%%%%%%%%%%%%%%%%%%%%%%%%
%%%%%%%%%%%%%%%%%%%%%%%%%%%%%%%%%%%%%%%%%%%%%%%%%%%%%%%%%%%%%%%%

\begin{numberedframe}{Overview}
  We briefly touch on peripheral issues issues to MPI.
\end{numberedframe}

\sectionframe{CMake}

\begin{numberedframe}{cmake}
  \label{sl:mpi-cmake}.
  MPI is discoverable by cmake.
\end{numberedframe}

\begin{c}
  \begin{numberedframe}{CMake for C}
    \lstinputlisting{makefiles/CMakeLists.mpi.c.txt}
  \end{numberedframe}
\end{c}

\begin{cxx}
  \begin{numberedframe}{CMake for C++}
    \lstinputlisting{makefiles/CMakeLists.mpi.cxx.txt}
  \end{numberedframe}
\end{cxx}

\begin{fortran}
\begin{numberedframe}{CMake for F90}
\lstinputlisting{makefiles/CMakeLists.mpi.f.txt}
\end{numberedframe}
\end{fortran}

\begin{fortran}
\begin{numberedframe}{CMake for F2008}
\lstinputlisting{makefiles/CMakeLists.mpi.f08.txt}
\end{numberedframe}
\end{fortran}

\sectionframe{Errors}

\begin{numberedframe}{Built-in handlers}
  \label{sl:errhandler}
Default: global termination.
\begin{lstlisting}
MPI_Comm_set_errhandler(MPI_COMM_WORLD,MPI_ERRORS_ARE_FATAL);
\end{lstlisting}

\begin{mpifour}
  Only terminate on communicator: \indexmpidef{MPI_ERRORS_ABORT}.
\end{mpifour}
  
Local handling: \indexmpishow{MPI_ERRORS_RETURN}:
\end{numberedframe}

\begin{numberedframe}{Handlers on specific classes}
Associate error handler with communicator:\\
  \indexmpishow{MPI_Comm_set_errhandler} \indexmpishow{MPI_Comm_get_errhandler}

  Other:
\begin{itemize}
\item  \indexmpishow{MPI_File_set_errhandler},
  \indexmpishow{MPI_File_call_errhandler},
\begin{mpifour}
  \indexmpishow{MPI_Session_set_errhandler},
  \indexmpishow{MPI_Session_call_errhandler},
\end{mpifour}
  \indexmpishow{MPI_Win_set_errhandler},
  \indexmpishow{MPI_Win_call_errhandler}.
\end{itemize}  
\end{numberedframe}

\begin{numberedframe}{Handling errors}
\begin{lstlisting}
char errtxt[MPI_MAX_ERROR_STRING];
int err = status.MPI_ERROR;
int len=MPI_MAX_ERROR_STRING;
MPI_Error_string(err,errtxt,&len);
printf("Waitall error: %d %s\n",err,errtxt);    
\end{lstlisting}
\end{numberedframe}

\begin{numberedframe}{Define new errors}
\begin{lstlisting}
int nonzero_code;
MPI_Add_error_code(nonzero_class,&nonzero_code);
MPI_Add_error_string(nonzero_code,"Attempting to send zero buffer");
\end{lstlisting}
\end{numberedframe}

\sectionframe{Performance measurement}

\begin{numberedframe}{Timers}
  MPI has a \indexterm{wall clock} timer: \indexmpishow{MPI_Wtime}
  which gives the number of seconds from a certain point in the past.

  The timer has a resolution of \indexmpishow{MPI_Wtick}

  Timers can be global
\begin{lstlisting}
int *v,flag;
MPI_Attr_get( comm, MPI_WTIME_IS_GLOBAL, &v, &flag );
if (mytid==0) printf("Time synchronized? %d->%d\n",flag,*v);
\end{lstlisting}
  but probably aren't.
\end{numberedframe}

\begin{numberedframe}{Example}
    \cverbatimsnippet{pingpong}
\end{numberedframe}

\begin{numberedframe}{Global timing}
  Processes don't start/end simultaneously. What does a timing result
  mean overall? Take average or maximum?

  Alternative:
\begin{lstlisting}
    MPI_Barrier(comm)
    t = MPI_Wtime();
    // something happens here
    MPI_Barrier(comm)
    t = MPI_Wtime()-t;
\end{lstlisting}
\end{numberedframe}

\begin{numberedframe}{Profiling}
  See other lecture: 
  MPIP, TAU, et cetera.
\end{numberedframe}

\begin{numberedframe}{Your own profiling interface}
  Every routine \lstinline{MPI_Something} calls a routine \lstinline{PMPI_Something} that 
  does the actual work. You can now write your \n{MPI_...} routine
  which calls \indexmpishow{PMPI_...}, and inserting your own profiling calls.

  \includegraphics[scale=.7]{graphics/pmpi}
\end{numberedframe}

\sectionframe{Programming for performance}

\begin{numberedframe}{Eager limit}
  \begin{itemize}
  \item Optimization for small messages: bypass rendez-vous protocol
    (slide~\ref{sl:rendezvous})
  \item Cross-over point: `Eager limit'.
  \item Force efficient messages by increasing the eager limit.
  \item Beware: decreasing payoff for large messages, and
  \item Beware: buffers for eager send eat into your available memory.
  \end{itemize}
\end{numberedframe}

\begin{numberedframe}{Eager limit setting}
    \begin{itemize}
    \item For Intel MPI: \n{I_MPI_EAGER_THRESHOLD}
    \item mvapich2: \n{MV2_IBA_EAGER_THRESHOLD}
    \item OpenMPI: \n{OpenMPI} the
      \texttt{--mca} options \indextermtt{btl_openib_eager_limit} and
      \indextermtt{btl_openib_rndv_eager_limit}.
    \end{itemize}
\end{numberedframe}

\begin{numberedframe}{Blocking versus non-blocking}
  \begin{itemize}
  \item Non-blocking sends \lstinline{MPI_Isend}~/ \lstinline{MPI_Irecv} can be more
    efficient than blocking
  \item Also: allow overlap computation/communication (latency hiding)
  \item However: can usually not be considered a replacement.
  \end{itemize}
\end{numberedframe}

\begin{numberedframe}{Progress}
  MPI is not magically active in the background, so latency hiding is
  not automatic. Same for passive target synchronization and
  non-blocking barrier completion.
  \begin{itemize}
  \item Dedicated communications processor or thread.\\
    This is implementation dependent; for instance,
    Intel MPI: \n{I_MPI_ASYNC_PROGRESS_...} variables.
  \item Force progress by occasional calls to a polling
    routine such as \indexmpishow{MPI_Iprobe}.
  \end{itemize}
\end{numberedframe}

\begin{numberedframe}{Persistent sends}
  If a communication between the same pair of processes, involving the
  same buffer, happens regularly, it is possible to set up a
  \indextermsub{persistent}{communication}.

  \begin{itemize}
  \item \lstinline{MPI_Send_init}
  \item \lstinline{MPI_Recv_init}
  \item \lstinline{MPI_Start}
  \end{itemize}

\end{numberedframe}

\begin{numberedframe}{Buffering}

  \begin{itemize}
  \item MPI has internal buffers: copying costs performance
  \item Use your own buffer:
    \begin{itemize}
    \item \lstinline{MPI_Buffer_attach}
    \item \lstinline{MPI_Bsend}
    \end{itemize}
  \item Copying is also a problem for derived datatypes.
  \end{itemize}
\end{numberedframe}

\begin{numberedframe}{Graph topology and neighborhood collectives}
  \begin{itemize}
  \item Mapping problem to architecture sometimes not trivial
  \item Load balancers: \indexterm{ParMetis}, \indexterm{Zoltan}
  \item Graph topologies: \lstinline{MPI_Dist_graph_adjacent}:\\
    allowed to reorder ranks for proximity
  \item Neighborhood collectives allow MPI to schedule optimally.
    \begin{itemize}
    \item \lstinline{MPI_Neighbor_allgather} (and \lstinline{MPI_Neighbor_allgather_v})
    \item \lstinline{MPI_Neighbor_alltoall}
    \end{itemize}
  \end{itemize}
\end{numberedframe}

\begin{numberedframe}{Network issues}
  Network contention means that
  \begin{itemize}
  \item Your messages can collide with other jobs
  \item messages within your job can collide
  \end{itemize}
\end{numberedframe}

\begin{numberedframe}{Output routing}
  \includegraphics[scale=.4]{oversubscription}
\end{numberedframe}

\begin{numberedframe}{Contention}
  \includegraphics[scale=.4]{contention}
\end{numberedframe}

\begin{numberedframe}{Offloading and onloading}

  \begin{itemize}
  \item Network cards can offer assistance
  \item Mellanox: off-loading\\
    limited repertoire of scenarios where it helps
  \item Intel disagrees: on-loading
  \item Either way, investigate the capabilities of your network.
  \end{itemize}
\end{numberedframe}

