% -*- latex -*-
%%%%%%%%%%%%%%%%%%%%%%%%%%%%%%%%%%%%%%%%%%%%%%%%%%%%%%%%%%%%%%%%
%%%%%%%%%%%%%%%%%%%%%%%%%%%%%%%%%%%%%%%%%%%%%%%%%%%%%%%%%%%%%%%%
%%%%
%%%% This text file is part of the source of 
%%%% `Parallel Computing'
%%%% by Victor Eijkhout, copyright 2012-2023
%%%%
%%%% mpi_course.tex : master file for an MPI course
%%%%
%%%%%%%%%%%%%%%%%%%%%%%%%%%%%%%%%%%%%%%%%%%%%%%%%%%%%%%%%%%%%%%%
%%%%%%%%%%%%%%%%%%%%%%%%%%%%%%%%%%%%%%%%%%%%%%%%%%%%%%%%%%%%%%%%

\documentclass[11pt,headernav]{beamer}

\beamertemplatenavigationsymbolsempty
\usetheme{Madrid}%{Montpellier}
\usecolortheme{seahorse}
\setcounter{tocdepth}{1}
\AtBeginSection[]
{
  \begin{frame}
    \frametitle{Table of Contents}
    \tableofcontents[currentsection]
  \end{frame}
}

\setbeamertemplate{footline}{\hskip1em Eijkhout: MPI intro\hfill
  \hbox to 0in {\hss \includegraphics[scale=.1]{tacclogonew}}%
  \hbox to 0in {\hss \arabic{page}\hskip 1in}}

\usepackage{multicol,multirow}
% custom arrays and tables
\usepackage{array} %,multirow,multicol}
\newcolumntype{R}{>{\hbox to 1.2em\bgroup\hss}{r}<{\egroup}}
\newcolumntype{T}{>{\hbox to 8em\bgroup}{c}<{\hss\egroup}}

\input slidemacs
\input coursemacs

%%
%% include definitions for the foundations 1-day course
%%
\def\Location{TACC Training, 2019}
\def\TitleExtra{, Foundations}

\excludecomment{online}

\excludecomment{imperial}
%\specialcomment{imperial}{\stepcounter{imperial}\def\CommentCutFile{imperial\arabic{imperial}.cut}}{}

\excludecomment{advanced}
\excludecomment{full}
\includecomment{onesided}

%%%%%%%%%%%%%%%%
%%%%%%%%%%%%%%%% Document
%%%%%%%%%%%%%%%%

\begin{document}
\parskip=10pt plus 5pt minus 3pt

\title{Tutorial on MPI programming\TitleExtra}
\author{Victor Eijkhout {\tt eijkhout@tacc.utexas.edu}}
\date{\Location}

\begin{frame}
  \titlepage
\end{frame}

  \begin{frame}{Justification}
    The MPI library is the main tool
    for parallel programming on a large scale.
    This course introduces the main concepts
    through lecturing and exercises.
  \end{frame}

  \Level 0 {The SPMD model}
  \input SPMD-slides

  \Level 0 {Collectives}
  \input Collective-slides

  \Level 0 {Point-to-point communication}
  \input PTP-slides

  \Level 0 {One-sided communication}
  \input Onesided-slides 

  \Level 0 {Derived Datatypes}
  \input Data-slides

  \Level 0 {Communicator manipulations}
  \input Subcomm-slides

  \Level 0 {MPI File I/O}
  \input MPIO-slides

  \Level 0 {Advanced collectives}
  \input Highercollective-slides

  \Level 0 {Process management}
  \input Spawn-slides

  \Level 0 {Process topologies}
  \input Graph-slides

  \Level 0 {Beginner exercise}

  \begin{exerciseframe}
    \input ex:serialsend
  \end{exerciseframe}

\end{document}

