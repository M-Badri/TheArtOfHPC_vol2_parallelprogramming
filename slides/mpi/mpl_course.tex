% -*- latex -*-
%%%%%%%%%%%%%%%%%%%%%%%%%%%%%%%%%%%%%%%%%%%%%%%%%%%%%%%%%%%%%%%%
%%%%%%%%%%%%%%%%%%%%%%%%%%%%%%%%%%%%%%%%%%%%%%%%%%%%%%%%%%%%%%%%
%%%%
%%%% This text file is part of the source of 
%%%% `Parallel Computing'
%%%% by Victor Eijkhout, copyright 2012-2023
%%%%
%%%% mpl_course.tex : course in MPL interface to MPI
%%%%
%%%%%%%%%%%%%%%%%%%%%%%%%%%%%%%%%%%%%%%%%%%%%%%%%%%%%%%%%%%%%%%%
%%%%%%%%%%%%%%%%%%%%%%%%%%%%%%%%%%%%%%%%%%%%%%%%%%%%%%%%%%%%%%%%

\documentclass[11pt,headernav]{beamer}

\beamertemplatenavigationsymbolsempty
\usetheme{Madrid}%{Montpellier}
\usecolortheme{seahorse}
\setcounter{tocdepth}{1}
%% \AtBeginSection[]
%% {
%%   \begin{frame}
%%     \frametitle{Table of Contents}
%%     \tableofcontents[currentsection]
%%   \end{frame}
%% }

\setbeamertemplate{footline}{\hskip1em Eijkhout: MPI course\hfill
  \hbox to 0in {\hss \includegraphics[scale=.1]{tacclogonew}}%
  \hbox to 0in {\hss \arabic{page}\hskip 1in}}

\usepackage{multicol,multirow}
% custom arrays and tables
\usepackage{array} %,multirow,multicol}
\newcolumntype{R}{>{\hbox to 1.2em\bgroup\hss}{r}<{\egroup}}
\newcolumntype{T}{>{\hbox to 8em\bgroup}{c}<{\hss\egroup}}

\usepackage{comment}
\input slides.inex
\input commonmacs
\input acromacs
\input slidemacs
\input coursemacs
\input snippetmacs

\includecomment{full}
\excludecomment{condensed}
\excludecomment{online}

\newcommand\referenceframe{\begingroup\catcode`\_=12 \referenceframett}
\begingroup\catcode`\_=12
\gdef\referenceframett#1{
  \begin{numberedframe}{Reference: \texttt{MPI_#1}}
    \small
    \verbatiminput{mplreference/MPI_#1.tex}
  \end{numberedframe}\endgroup}
\endgroup

\includecomment{utonly}

\includecomment{onesided}
\includecomment{advanced}
\includecomment{foundations}

\def\Location{}% redefine in the inex file
\def\courseyear{2021}
\def\Location{TACC APP institute MPI training \courseyear}
\def\TitleExtra{}

\input lang

%%%%%%%%%%%%%%%%
%%%%%%%%%%%%%%%% Document
%%%%%%%%%%%%%%%%

\begin{document}
\parskip=10pt plus 5pt minus 3pt

\title{Tutorial on the MPL interface to MPI}
\author{Victor Eijkhout {\tt eijkhout@tacc.utexas.edu}}
\date{\Location}

\begin{frame}
  \titlepage
\end{frame}

\begin{frame}{Justification}
  While the C API to MPI is usable from C++, it feels very unidiomatic
  for that language.
  \acf{MPL} is a modern C++11 interface to MPI.
  It is both idiomatic and elegant, simplifying many calling sequences.
\end{frame}

\Level 0 {Basics}

\begin{numberedframe}{Environment}
  For doing the exercises:
\begin{verbatim}
module load mpl
\end{verbatim}
which defines \n{TACC_MPL_INC}, \n{TACC_MPL_DIR}
\end{numberedframe}

\begin{numberedframe}{Header file}
  \input{mplnote-header-file.cut}
\end{numberedframe}
\begin{numberedframe}{Init, finalize}
  \input{mplnote-init,-finalize.cut}
\end{numberedframe}
\begin{numberedframe}{World communicator}
  \input{mplnote-world-communicator.cut}
\end{numberedframe}
\begin{numberedframe}{Processor name}
  \input{mplnote-processor-name.cut}
\end{numberedframe}
\begin{numberedframe}{Rank and size}
  \input{mplnote-rank-and-size.cut}
\end{numberedframe}
\referenceframe{Comm_size}
\referenceframe{Comm_rank}

\begin{numberedframe}{Timing}
  \input{mplnote-timing.cut}
\end{numberedframe}
\begin{numberedframe}{Predefined communicators}
  \input{mplnote-predefined-communicators.cut}
\end{numberedframe}
\begin{numberedframe}{Communicator copying}
  \input{mplnote-communicator-copying.cut}
\end{numberedframe}
\begin{numberedframe}{Communicator duplication}
  \input{mplnote-communicator-duplication.cut}
\end{numberedframe}
\begin{numberedframe}{Communicator passing}
  \input{mplnote-communicator-passing.cut}
\end{numberedframe}

\Level 0 {Collectives}

\begin{frame}{Introduction}
  Collectives have many polymorphic variants, for instance for `in place',
  and buffer handling.

  Operators are handled through functors.
\end{frame}
\begin{numberedframe}{Scalar buffers}
  \input{mplnote-scalar-buffers.cut}
\end{numberedframe}
\begin{numberedframe}{Vector buffers}
  \input{mplnote-vector-buffers.cut}
\end{numberedframe}
\begin{numberedframe}{Iterator buffers}
  \input{mplnote-iterator-buffers.cut}
\end{numberedframe}

\begin{numberedframe}{Reduction operator}
  % this is really a reduce slide
  \input{mplnote-reduction-operator.cut}
\end{numberedframe}
\referenceframe{Allreduce}
\referenceframe{Reduce}
\begin{numberedframe}{Broadcast}
  \input{mplnote-broadcast.cut}
\end{numberedframe}
\referenceframe{Bcast}
\begin{numberedframe}{Gather scatter}
  \input{mplnote-gather-scatter.cut}
\end{numberedframe}
\begin{numberedframe}{Reduce on non-root processes}
  \input{mplnote-reduce-on-non-root.cut}
\end{numberedframe}
\begin{numberedframe}{Gather on non-root}
  \input{mplnote-gather-on-nonroot.cut}
\end{numberedframe}
\referenceframe{Gather}
\begin{numberedframe}{Reduce in place}
  \footnotesize
  \input{mplnote-reduce-in-place.cut}
\end{numberedframe}
%% \begin{numberedframe}{Rooted reduce}
%%   \input{mplnote-rooted-reduce.cut}
%% \end{numberedframe}
\begin{numberedframe}{Layouts for gatherv}
  \input{mplnote-layouts-for-gatherv.cut}
\end{numberedframe}
\begin{numberedframe}{Scan operations}
  \input{mplnote-scan-operations.cut}
\end{numberedframe}

\begin{exerciseframe}[scangather]
  \input ex:scanprint
\end{exerciseframe}
\begin{exerciseframe}[scangather]
  \input ex:scangather
\end{exerciseframe}

\begin{numberedframe}{Operators}
  \input{mplnote-operators.cut}
\end{numberedframe}
\begin{numberedframe}{User defined operators}
  \input{mplnote-user-defined-operators.cut}
\end{numberedframe}
\begin{numberedframe}{Lambda reduction operators}
  \input{mplnote-lambda-operator.cut}
\end{numberedframe}
\begin{numberedframe}{Nonblocking collectives}
  \input{mplnote-nonblocking-collectives.cut}
\end{numberedframe}

\Level 0 {Point-to-point communication}

\begin{numberedframe}{Buffer type safety}
  \input{mplnote-buffer-type-safety.cut}
\end{numberedframe}
\begin{numberedframe}{Blocking send and receive}
  \input{mplnote-blocking-send-and-receive.cut}
\end{numberedframe}
\begin{numberedframe}{Sending arrays}
  \input{mplnote-sending-arrays.cut}
\end{numberedframe}
\referenceframe{Send}
\referenceframe{Recv}
\begin{numberedframe}{Tag types}
  \input{mplnote-tag-types.cut}
\end{numberedframe}
\begin{numberedframe}{Message tag}
  \input{mplnote-message-tag.cut}
\end{numberedframe}
\begin{numberedframe}{Any source}
  \input{mplnote-any-source.cut}
\end{numberedframe}
\begin{numberedframe}{Send-recv call}
  \input{mplnote-send-recv-call.cut}
\end{numberedframe}
\begin{numberedframe}{Status object}
  \input{mplnote-status-object.cut}
\end{numberedframe}
\begin{numberedframe}{Status source querying}
  \input{mplnote-status-source-querying.cut}
\end{numberedframe}
\begin{numberedframe}{Receive count}
  \input{mplnote-receive-count.cut}
\end{numberedframe}

\begin{numberedframe}{Requests from nonblocking calls}
  \input{mplnote-requests-from-nonblocking-calls.cut}
\end{numberedframe}
\begin{numberedframe}{Request pools}
  \input{mplnote-request-pools.cut}
\end{numberedframe}
\begin{numberedframe}{Wait any}
  \input{mplnote-wait-any.cut}
\end{numberedframe}
\begin{exerciseframe}[setdiff]
  \input ex:setdiff
\end{exerciseframe}

\begin{numberedframe}{Buffered send}
  \input{mplnote-buffered-send.cut}
\end{numberedframe}
\begin{numberedframe}{Buffer attach and detach}
  \input{mplnote-buffer-attach-and-detach.cut}
\end{numberedframe}
\begin{numberedframe}{Persistent requests}
  \input{mplnote-persistent-requests.cut}
\end{numberedframe}

\Level 0 {Derived Datatypes}

\begin{numberedframe}{Datatype handling}
  \input{mplnote-datatype-handling.cut}
\end{numberedframe}
\begin{numberedframe}{Native MPI datatypes}
  \input{mplnote-native-mpi-data-types.cut}
\end{numberedframe}
\begin{numberedframe}{Derived type handling}
  \input{mplnote-derived-type-handling.cut}
\end{numberedframe}
\begin{numberedframe}{Contiguous type}
  \input{mplnote-contiguous-type.cut}
\end{numberedframe}
\begin{numberedframe}{Contiguous composing of types}
  \input{mplnote-contiguous-composing.cut}
\end{numberedframe}
\begin{numberedframe}{Vector type}
  \input{mplnote-vector-type.cut}
\end{numberedframe}
\begin{numberedframe}{Iterator buffers}
  \input{mplnote-iterator-buffers.cut}
\end{numberedframe}
\begin{numberedframe}{Iterator layout}
  \input{mplnote-iterator-layout.cut}
\end{numberedframe}
\begin{numberedframe}{Subarray layout}
  \input{mplnote-subarray-layout.cut}
\end{numberedframe}
\begin{numberedframe}{Indexed type}
  \input{mplnote-indexed-type.cut}
\end{numberedframe}
\begin{numberedframe}{Indexed block type}
  \input{mplnote-indexed-block-type.cut}
\end{numberedframe}
\begin{numberedframe}{Struct type scalar}
  \input{mplnote-struct-type-scalar.cut}
\end{numberedframe}
\begin{numberedframe}{Struct type general}
  \input{mplnote-struct-type-general.cut}
\end{numberedframe}
\begin{numberedframe}{Extent resizing}
  \input{mplnote-extent-resizing.cut}  
\end{numberedframe}

\Level 0 {Communicator manipulations}

\begin{numberedframe}{Communicator errhandler}
  \input{mplnote-communicator-errhandler.cut}
\end{numberedframe}
\begin{numberedframe}{Communicator splitting}
  \input{mplnote-communicator-splitting.cut}
\end{numberedframe}
\begin{numberedframe}{Split by shared memory}
  \input{mplnote-split-by-shared-memory.cut}
\end{numberedframe}

\Level 0 {Process topologies}

%% \begin{numberedframe}{Distributed graph creation}
%%   \input{mplnote-distributed-graph-creation.cut}
%% \end{numberedframe}
\begin{numberedframe}{Graph communicators}
  \input{mplnote-graph-communicators.cut}
\end{numberedframe}
\begin{numberedframe}{Graph communicator querying}
  \input{mplnote-graph-communicator-querying.cut}
\end{numberedframe}

\Level 0 {Other}

\begin{numberedframe}{Timing}
  \input{mplnote-timing.cut}
\end{numberedframe}

\begin{numberedframe}{Threading support}
  \input{mplnote-threading-support.cut}
\end{numberedframe}

\begin{frame}{Missing from MPL}
  MPL is not a full MPI implementation
  \begin{itemize}
  \item File I/O
  \item One-sided communication
  \item Shared memory
  \item Process management
  \end{itemize}
\end{frame}

\end{document}

