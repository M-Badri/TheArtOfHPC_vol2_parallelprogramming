% -*- latex -*-
%%%%%%%%%%%%%%%%%%%%%%%%%%%%%%%%%%%%%%%%%%%%%%%%%%%%%%%%%%%%%%%%
%%%%%%%%%%%%%%%%%%%%%%%%%%%%%%%%%%%%%%%%%%%%%%%%%%%%%%%%%%%%%%%%
%%%%
%%%% This text file is part of the lecture slides for
%%%% `Parallel Computing'
%%%% by Victor Eijkhout, copyright 2022
%%%%
%%%% Bigdata-slides.tex : slides about messages beyond 2G
%%%%
%%%%%%%%%%%%%%%%%%%%%%%%%%%%%%%%%%%%%%%%%%%%%%%%%%%%%%%%%%%%%%%%
%%%%%%%%%%%%%%%%%%%%%%%%%%%%%%%%%%%%%%%%%%%%%%%%%%%%%%%%%%%%%%%%

\begin{numberedframe}{Overview}
  This section discusses big messages.

  Commands learned:
  \begin{itemize}
  \item \indexmpishow{MPI_Send_c}, \indexmpishow{MPI_Allreduce_c},
    \indexmpishow{MPI_Get_count_c}~etc. (MPI-4)
  \end{itemize}
\end{numberedframe}

\begin{numberedframe}{The problem with large messages}
  \begin{itemize}
  \item There is no problem allocating large buffers:
\begin{lstlisting}
size_t bigsize = 1<<33;
double *buffer =
    (double*) malloc(bigsize*sizeof(double));
\end{lstlisting}
\item But you can not tell MPI how big the buffer is:
\begin{lstlisting}
MPI_Send(buffer,bigsize,MPI_DOUBLE,...) // WRONG
\end{lstlisting}
because the size argument has to be \lstinline{int}.
  \end{itemize}
\end{numberedframe}

\begin{numberedframe}{MPI 3 count type}
  Count type since MPI 3\\
  C:
\begin{lstlisting}
MPI_Count count;
\end{lstlisting}
Fortran:
\lstset{language=Fortran}
\begin{lstlisting}
Integer(kind=MPI_COUNT_KIND) :: count
\end{lstlisting}
\lstset{language=C}
Big enough for
\begin{itemize}
\item
  \lstinline{int};
\item \indexmpishow{MPI_Aint}, used in one-sided;
\item \indexmpishow{MPI_Offset}, used in file I/O.
\end{itemize}
However, this type could not be used in MPI-3 to describe send buffers.
\end{numberedframe}

\begin{numberedframe}{MPI 4 large count routines}
C: routines with \lstinline+_c+ suffix
\begin{lstlisting}
MPI_Count count;
MPI_Send_c( buff,count,MPI_INT, ... );
\end{lstlisting}
\lstset{language=Fortran}
also \indexmpishow{MPI_Reduce_c}, \indexmpishow{MPI_Get_c},~\ldots
(some 190 routines in all)

Fortran: polymorphism rules
\begin{lstlisting}
Integer(kind=MPI_COUNT_KIND) :: count
call MPI_Send( buff,count, MPI_INTEGER, ... )
\end{lstlisting}
\lstset{language=C}
\end{numberedframe}

\begin{numberedframe}{Big count example}
  \cverbatimsnippet{bigpingpong}
\end{numberedframe}

\begin{numberedframe}{Same in F08}
  \fverbatimsnippet{bigdataf}  
\end{numberedframe}

\protoslide{MPI_Send}

\begin{numberedframe}{MPI 4 large count querying}
C: 
\begin{lstlisting}
MPI_Count count;
MPI_Get_count_c( &status,MPI_INT, &count );
MPI_Get_elements_c( &status,MPI_INT, &count );
\end{lstlisting}
\lstset{language=Fortran}
Fortran: 
\begin{lstlisting}
Integer(kind=MPI_COUNT_KIND) :: count
call MPI_Get_count( status,MPI_INTEGER,count )
call MPI_Get_elements( status,MPI_INTEGER,count )
\end{lstlisting}
\lstset{language=C}
\end{numberedframe}

\begin{numberedframe}{Large int counting}
  MPI-3 mechanism, deprecated in MPI-4.1:\\
  send a number of contiguous types:

  \cverbatimsnippet{bigvectorptp}

  By composing types you can make a `big type'. Use\\
  \indexmpishow{MPI_Type_get_extent_x},
  \indexmpishow{MPI_Type_get_true_extent_x},
   \lstinline{MPI_Get_elements_x}\\
  to query.

  \cverbatimsnippet{bigvectorq}
\end{numberedframe}

%\protoslide{MPI_Get_elements_x}

\begin{comment}
  \begin{numberedframe}{For your amusement}
    What do you get if you print the following:
    \cverbatimsnippet{compsizet}
    ?
  \end{numberedframe}
\end{comment}

\begin{comment}
\begin{python}
\begin{numberedframe}{Big data in python}
  The \n{mpi4py} interface uses Python integers,\\
  which are 8 bytes. So the \n{_x} routines are not needed.
\end{numberedframe}
\end{python}
\end{comment}

\endinput

\begin{numberedframe}{}
\begin{lstlisting}
\end{lstlisting}
\end{numberedframe}

